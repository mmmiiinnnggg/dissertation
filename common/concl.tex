%% Согласно ГОСТ Р 7.0.11-2011:
%% 5.3.3 В заключении диссертации излагают итоги выполненного исследования, рекомендации, перспективы дальнейшей разработки темы.
%% 9.2.3 В заключении автореферата диссертации излагают итоги данного исследования, рекомендации и перспективы дальнейшей разработки темы.
\begin{enumerate}
    \item На основе анализа \ldots
    \item Численные исследования показали, что \ldots
    \item Математическое моделирование показало \ldots
    \item Для выполнения поставленных задач был создан \ldots
\end{enumerate}


% В настоящей диссертационной работе решена актуальная научная задача разработки и исследования методов автоматизированного поиска и распознавания иероглифических символов в оцифрованных исторических архивах. В отличие от традиционных подходов, ориентированных на статистический анализ пиксельных представлений, в данной работе предложен и реализован комплексный подход, основанный на аппарате непрерывной морфологии бинарных изображений.

% Основные научные и практические результаты работы заключаются в следующем:

% Разработана модель точечного признакового описания иероглифов, базирующаяся на скелетном представлении и понятии «созвездия» признаков. Использование непрерывной морфологии позволило перейти от дискретной сетки пикселей к компактному геометрическому описанию структуры иероглифа. Это обеспечило инвариантность описания к вариациям начертания, типичным для рукописных и ксилографических памятников.

% Предложен новый метод вычисления меры сходства между иероглифами, основанный на решении задачи о назначениях (взвешенном сопоставлении точек созвездий). Разработанная целевая функция учитывает не только геометрическое положение ключевых точек скелета, но и их локальные морфологические характеристики (радиусы вписанных кругов). Такой подход позволил существенно повысить точность ранжирования при поиске по образцу (query-by-example).

% Реализован программный комплекс для анализа и поиска в больших массивах изображений исторических архивов. В ходе экспериментальной апробации на реальных данных подтверждена эффективность предложенных алгоритмов. Установлено, что разработанный метод поиска обеспечивает высокую вероятность попадания релевантных символов в топ-5\% и топ-10\% списка кандидатов, что является критически важным для работы эксперта-ассессора в диалоговом режиме.
