
{\actuality}
% общее введение
Оптическое распознавание символов – это технология, которая преобразует цифровые изображения рукописного или печатного текста в машинный текст для дальнейшего хранения и обработки. В последние годы были разработаны передовые методы работы с изображениями печатных текстов, но лишь небольшая часть исследований посвящена распознаванию текстов из рукописных архивов. В то же время, древние иероглифы в рукописных текстах имеют огромную исследовательскую ценность как гуманитарное наследие, фиксирующее историческую культуру, язык, развитие науки.

% задача поиска
Поиск информации и навигация по запросам в огромном количестве древних документов всегда остаются одной из центральных и актуальных проблем в исследовании древних текстов. Задача поиска состоит в нахождении изображения отдельного заданного иероглифа, называемого запросом, в большом множестве изображений иероглифов, называемом файлом. Целью является локализация в файле всех иероглифов, совпадающих с запросом. В распознавании образов задача использования одного образца для распознавания в литературе часто называется \textit{однократным обучением}.

Применение современных методов распознавания, использующих машинное обучение и глубокие нейронные сети, сталкивается с большими трудностями при подготовке обучающих данных по древнекитайским иероглифам:

\begin{itemize}[label=•]
    \item \textbf{Огромный словарь письменности}. Для обучения требуется очень большой объем данных, поскольку очень велико число символов древнекитайской письменности. В настоящее время общее число уникальных иероглифов, встречающихся в исторических источниках, превышает $6\times 10^4$, включая устаревшие и редко используемые.
    \item \textbf{Сложность и вариативность документов}. Сложность структуры китайских иероглифов в сочетании с вариативностью, обусловленной индивидуальными стилями почерка, деградацией бумаги и неравномерностью штрихов создает существенные трудности для традиционных методов извлечения и сопоставления признаков. Сбор обучающей выборки, захватывающей все виды и стили документов практически невозможно.
    \item \textbf{Высокая трудоёмкость аннотирования}. Весьма велики трудозатраты на аннотирование символов даже для специалиста в области древних иероглифов. На распознавание символов и извлечение ключевой информации из нескольких страниц древних рукописей тратятся несколько часов или даже несколько дней.
\end{itemize}

Кроме того, как для обучения, так и для практического применения требуются современные ускорители, стоимость их использования относительно высока. Современные нейронные сети не дают возможности интерпретировать лежащий в основе распознавания процесс. Но понимание мотивов формируемого в сети решения очень важно для исследователей, поскольку это помогает лучше понять структуру символов, облегчает адаптацию к другим языкам и пути совершенствования алгоритма распознавания.

Таким образом, существует острая потребность в \textbf{интерпретируемых}, \textbf{вычислительно устойчивых} и \textbf{обучением-независимых} методах сравнения графических форм иероглифов для задач поиска и навигации в архивных коллекциях, которые могут сильно упростить работу исторических исследователей при минимальным требовании к ресурсам, и этот факт определяет актуальность данного исследования.

% \ifsynopsis
%     Этот абзац появляется только в~автореферате.
%     Для формирования блоков, которые будут обрабатываться только в~автореферате,
%     заведена проверка условия \verb!\!\verb!ifsynopsis!.
%     Значение условия задаётся в~основном файле документа (\verb!synopsis.tex! для
%     автореферата).
% \else
%     Этот абзац появляется только в~диссертации.
%     Через проверку условия \verb!\!\verb!ifsynopsis!, задаваемого в~основном файле
%     документа (\verb!dissertation.tex! для диссертации), можно сделать новую
%     команду, обеспечивающую появление цитаты в~диссертации, но~не~в~автореферате.
% \fi

% {\progress}
% Этот раздел должен быть отдельным структурным элементом по
% ГОСТ, но он, как правило, включается в описание актуальности
% темы. Нужен он отдельным структурынм элемементом или нет ---
% смотрите другие диссертации вашего совета, скорее всего не нужен.

{\aim} диссертационной работы является разработка математического и программного обеспечения для решения задач поиска и навигации в массивных древнекитайских рукописях с использованием методов непрерывных морфологических моделей и методов определения меры сходства.

Объектом исследования диссертационной работы являются модели построения точечных признаковых описаниях и модели вычисления меры сходства между признаками изображений иероглифов. Предметом исследования диссертационной работы является разработка алгоритмов построения моделей генерации точечных признаков и оптимизационных моделей получения меры схожести, применимых к древнекитайским рукописным документам.

Для~достижения поставленной цели необходимо было решить следующие {\tasks}:
\begin{enumerate}[beginpenalty=10000] % https://tex.stackexchange.com/a/476052/104425
    \item Формирование состава задач автоматизации работы с цифровыми архивами рукописных иероглифических документов.
    \item Структурный анализ рукописных китайских иероглифов и получение признакового описания с использованием медиального представления формы многоугольной фигуры, полученное на основе диаграммы Вороного.
    \item Построение меры сходства рукописных изображений иероглифов и разработка алгоритма его вычисления.
    \item Практическая реализация разработанных методов построения моделей для контекстного поиска в больших рукописных архивах по ключевым словам и проведение экспериментов для проверки корректности полученных результатов.
\end{enumerate}

% соответствие паспорта специальности
Диссертация соответствует специальности 1.2.1 <<Искусственный интеллект и машинное обучение>> в части направления разработки методов, алгоритмов и создание систем искусственного интеллекта и машинного обучения для обработки и анализа текстов на естественном языке, для изображений, речи, биомедицины и других специальных видов данных, поскольку целью данной работы является исследование, разработка методов для анализа и обработки изображений древнекитайских рукописей и текстов естественного языка, содержащих в документах рукописей.

{\novelty}
\begin{enumerate}[beginpenalty=10000] % https://tex.stackexchange.com/a/476052/104425
    \item Предложены новые математические модели, позволяющие эффективно анализировать форму древного иероглифа на основе непрерывных морфологиях, и представляют признаковое описание изображения иероглифа в виде конечного множества критичных точек, характеризующих топологическое и геометрическое свойство иероглифа.
    \item Предложены модели для сравнения иероглифических точечных признаков, полученных предложенными моделями построения точечных признаков, по принципу сопоставления соответствия пар критичных точек.
    \item Разработана оптимизационная модель через задачу линейного программирования для вычисления количественной меры сходства между иероглифами.
\end{enumerate}

{\influence}
Научная значимость заключается в разработке методов построения признакового описания древнекитайского рукописного иероглифа в виде конечного набора критичных точек через медиальное представления формы бинарного изображения, также в разработке методов сравнения форм на основе оптимизационной модели линейного программирования, используя построенные точечные признаки иероглифов. Предложенные подходы позволяют построить представление иероглифов прозрачным математическим аппаратом с вычислительно эффективной процедурой, в то же время семантично достаточно для последующего сравнения форм иероглифов по принципу сопоставления точек.

Практическая значимость состоит в детальной программной реализации всех предложенных методов, а также их приложений для решения различных прикладных задач, связанных с поиском информации в архивных документах. Апробация реализации разработанных методов проводилась на реальных данных отсканированных древнекитайских документов. По результатам экспериментального исследования, предложенные методы сравнимы по качеству с существующими методами распознавания иероглифических изображений, но имеют минимальное требование к ресурсам поскольку не нуждаются в тяжелой процедуре обучения и сбора бучающей выборки, требующего значительное человеческое усилие, и преимущество интерпретируемости для облегчения адаптации к другим языкам и задачам распознавания.

{\methods} При получении основных результатов диссертационной работы использовались методы обработки и анализа изображений, методы вычислительной геометрии, теория графов, методы оптимизации. Работа носит экспериментально-теоретический характер. Разработка программного кода велась на языке Python с использованием библиотеки скелетизации, разработанной Л.М.
Местецким. Эксперименты проводились на модельных данных и открытых
базах изображений иероглифов.

{\defpositions}
\begin{enumerate}[beginpenalty=10000] % https://tex.stackexchange.com/a/476052/104425
    \item Методы построения признакового описания для древнекитайского иероглифа в виде конечного набора критичных точек на основе медиального представления и непрерывных морфологий. Предложен метода выбора критичных точек с учетом максимальной кривизной.
    \item Методы построения оптимизационной модели для сравнения точечных признаков иероглифа. Разработаны методы получения меры близости путем решения задачи о назначениях, которые могут быть реализованы эффективной процедурой вычисления.
    \item Обоснование работоспособности предложенных методов путём реализации программного комплекса. Экспериментально доказано, что разработанные методы не уступают по качеству существующим современным методам с использованием нейронных сетей, при этом обладают преимуществом отсутствия требования обучения на гиганской обучающей выборки, низкого запроса к вычислительным ресурсам и интерпретируемости алгоритма.
\end{enumerate}

Все результаты, выносимые на защиту, получены автором самостоятельно под руководством научного руководители Л.М.Местецкого.
% В папке Documents можно ознакомиться с решением совета из Томского~ГУ
% (в~файле \verb+Def_positions.pdf+), где обоснованно даются рекомендации
% по~формулировкам защищаемых положений.

{\reliability} полученных результатов обеспечивается проведенными экспериментами, корректным тестированием разработанных решений, публикациями в рецензируемых журналах и апробацией на российских и международных конференциях.

{\probation}
Основные результаты работы докладывались~на:
\begin{itemize}
    \item Международная научная конференция студентов, аспирантов и молодых учёных <<Ломоносов-2025>> (Россия, Москва, 2025).
    \item 6th International workshop on Photogrammetric techniques for environmental and infrastructure monitoring, Biometry and Biomedicine (Россия, Москва, 2025).
    \item 3rd International Conference on Machine Intelligence and Digital Applications (Китай, Сиань, 2026).
    \item ??? Международная научная конференция студентов, аспирантов и молодых учёных <<Ломоносов-2026>> (Россия, Москва, 2026).
\end{itemize}

Автор принимал активное участие в работе научного семинара Л.М.Местецкого «Непрерывные морфологические модели и алгоритмы» (факультет вычислительной математики и кибернетики МГУ).

    {\contribution} Автор принимал активное участие в выполнении основного объема теоретических и экспериментальных исследований, а также в разработке реализации разработанных методов. Подготовка части материалов к публикации проводилась совместно с соавторами, причем вклад диссертанта был определяющим. Диссертационное исследование является самостоятельлным и законченным трудом автора.

{\publications} Основные результаты по теме диссертации изложены в~XX~печатных изданиях, X из которых изданы в журналах, рекомендованных ВАК, X "--- в тезисах докладов.

% \ifnumequal{\value{bibliosel}}{0}
% {%%% Встроенная реализация с загрузкой файла через движок bibtex8. (При желании, внутри можно использовать обычные ссылки, наподобие `\cite{vakbib1,vakbib2}`).
%     {\publications} Основные результаты по теме диссертации изложены
%     в~XX~печатных изданиях,
%     X из которых изданы в журналах, рекомендованных ВАК,
%     X "--- в тезисах докладов.
% }%
% {%%% Реализация пакетом biblatex через движок biber
%     \begin{refsection}[bl-author, bl-registered]
%         % Это refsection=1.
%         % Процитированные здесь работы:
%         %  * подсчитываются, для автоматического составления фразы "Основные результаты ..."
%         %  * попадают в авторскую библиографию, при usefootcite==0 и стиле `\insertbiblioauthor` или `\insertbiblioauthorgrouped`
%         %  * нумеруются там в зависимости от порядка команд `\printbibliography` в этом разделе.
%         %  * при использовании `\insertbiblioauthorgrouped`, порядок команд `\printbibliography` в нём должен быть тем же (см. biblio/biblatex.tex)
%         %
%         % Невидимый библиографический список для подсчёта количества публикаций:
%         \phantom{\printbibliography[heading=nobibheading, section=1, env=countauthorvak,          keyword=biblioauthorvak]%
%             \printbibliography[heading=nobibheading, section=1, env=countauthorwos,          keyword=biblioauthorwos]%
%             \printbibliography[heading=nobibheading, section=1, env=countauthorscopus,       keyword=biblioauthorscopus]%
%             \printbibliography[heading=nobibheading, section=1, env=countauthorconf,         keyword=biblioauthorconf]%
%             \printbibliography[heading=nobibheading, section=1, env=countauthorother,        keyword=biblioauthorother]%
%             \printbibliography[heading=nobibheading, section=1, env=countregistered,         keyword=biblioregistered]%
%             \printbibliography[heading=nobibheading, section=1, env=countauthorpatent,       keyword=biblioauthorpatent]%
%             \printbibliography[heading=nobibheading, section=1, env=countauthorprogram,      keyword=biblioauthorprogram]%
%             \printbibliography[heading=nobibheading, section=1, env=countauthor,             keyword=biblioauthor]%
%             \printbibliography[heading=nobibheading, section=1, env=countauthorvakscopuswos, filter=vakscopuswos]%
%             \printbibliography[heading=nobibheading, section=1, env=countauthorscopuswos,    filter=scopuswos]}%
%         %
%         \nocite{*}%
%         %
%         {\publications} Основные результаты по теме диссертации изложены в~\arabic{citeauthor}~печатных изданиях,
%         \arabic{citeauthorvak} из которых изданы в журналах, рекомендованных ВАК%
%         \ifnum \value{citeauthorscopuswos}>0%
%             , \arabic{citeauthorscopuswos} "--- в~периодических научных журналах, индексируемых Web of~Science и Scopus%
%         \fi%
%         \ifnum \value{citeauthorconf}>0%
%             , \arabic{citeauthorconf} "--- в~тезисах докладов.
%         \else%
%             .
%         \fi%
%         \ifnum \value{citeregistered}=1%
%             \ifnum \value{citeauthorpatent}=1%
%                 Зарегистрирован \arabic{citeauthorpatent} патент.
%             \fi%
%             \ifnum \value{citeauthorprogram}=1%
%                 Зарегистрирована \arabic{citeauthorprogram} программа для ЭВМ.
%             \fi%
%         \fi%
%         \ifnum \value{citeregistered}>1%
%             Зарегистрированы\ %
%             \ifnum \value{citeauthorpatent}>0%
%                 \formbytotal{citeauthorpatent}{патент}{}{а}{}%
%                 \ifnum \value{citeauthorprogram}=0 . \else \ и~\fi%
%             \fi%
%             \ifnum \value{citeauthorprogram}>0%
%                 \formbytotal{citeauthorprogram}{программ}{а}{ы}{} для ЭВМ.
%             \fi%
%         \fi%
%         % К публикациям, в которых излагаются основные научные результаты диссертации на соискание учёной
%         % степени, в рецензируемых изданиях приравниваются патенты на изобретения, патенты (свидетельства) на
%         % полезную модель, патенты на промышленный образец, патенты на селекционные достижения, свидетельства
%         % на программу для электронных вычислительных машин, базу данных, топологию интегральных микросхем,
%         % зарегистрированные в установленном порядке.(в ред. Постановления Правительства РФ от 21.04.2016 N 335)
%     \end{refsection}%
%     \begin{refsection}[bl-author, bl-registered]
%         % Это refsection=2.
%         % Процитированные здесь работы:
%         %  * попадают в авторскую библиографию, при usefootcite==0 и стиле `\insertbiblioauthorimportant`.
%         %  * ни на что не влияют в противном случае
%         \nocite{vakbib2}%vak
%         \nocite{patbib1}%patent
%         \nocite{progbib1}%program
%         \nocite{bib1}%other
%         \nocite{confbib1}%conf
%     \end{refsection}%
%     %
%     % Всё, что вне этих двух refsection, это refsection=0,
%     %  * для диссертации - это нормальные ссылки, попадающие в обычную библиографию
%     %  * для автореферата:
%     %     * при usefootcite==0, ссылка корректно сработает только для источника из `external.bib`. Для своих работ --- напечатает "[0]" (и даже Warning не вылезет).
%     %     * при usefootcite==1, ссылка сработает нормально. В авторской библиографии будут только процитированные в refsection=0 работы.
% }

% При использовании пакета \verb!biblatex! будут подсчитаны все работы, добавленные
% в файл \verb!biblio/author.bib!. Для правильного подсчёта работ в~различных
% системах цитирования требуется использовать поля:
% \begin{itemize}
%     \item \texttt{authorvak} если публикация индексирована ВАК,
%     \item \texttt{authorscopus} если публикация индексирована Scopus,
%     \item \texttt{authorwos} если публикация индексирована Web of Science,
%     \item \texttt{authorconf} для докладов конференций,
%     \item \texttt{authorpatent} для патентов,
%     \item \texttt{authorprogram} для зарегистрированных программ для ЭВМ,
%     \item \texttt{authorother} для других публикаций.
% \end{itemize}
% Для подсчёта используются счётчики:
% \begin{itemize}
%     \item \texttt{citeauthorvak} для работ, индексируемых ВАК,
%     \item \texttt{citeauthorscopus} для работ, индексируемых Scopus,
%     \item \texttt{citeauthorwos} для работ, индексируемых Web of Science,
%     \item \texttt{citeauthorvakscopuswos} для работ, индексируемых одной из трёх баз,
%     \item \texttt{citeauthorscopuswos} для работ, индексируемых Scopus или Web of~Science,
%     \item \texttt{citeauthorconf} для докладов на конференциях,
%     \item \texttt{citeauthorother} для остальных работ,
%     \item \texttt{citeauthorpatent} для патентов,
%     \item \texttt{citeauthorprogram} для зарегистрированных программ для ЭВМ,
%     \item \texttt{citeauthor} для суммарного количества работ.
% \end{itemize}
% % Счётчик \texttt{citeexternal} используется для подсчёта процитированных публикаций;
% % \texttt{citeregistered} "--- для подсчёта суммарного количества патентов и программ для ЭВМ.

% Для добавления в список публикаций автора работ, которые не были процитированы в
% автореферате, требуется их~перечислить с использованием команды \verb!\nocite! в
% \verb!Synopsis/content.tex!.
