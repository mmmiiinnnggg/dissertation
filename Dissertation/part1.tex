\chapter{Обзор существующих методов и математических средств}\label{ch:ch1}

\section{Обзор методов распознавание иероглифов}\label{sec:ch1/sec1}
% Из PRIA

% OCR 总体背景(打印文本 → 成熟)
% 问题聚焦:手写 + 历史文献
% 难点来源(图像质量 + 字形多样性)
% 方法分类(传统 / 深度学习 / 混合)
% 关键瓶颈:超大字表 + 数据稀缺
% 引出 One-shot Learning 作为理论框架
% 特征表示问题 → 骨架图
% 自然过渡到你自己的方法

Распознавание текстов документов является одной из ключевых задач в области хранения, обработки и поиска информации. В последние годы для анализа печатных изображений документов были разработаны передовые методы, которые позволяют достигать высокого качества распознавания и уже успешно применяются в различных системах автоматизации рутинных процессов \cite{reisswig2019chargrid, smith2007overview, du2009pp, li2023trocr, xu2020layoutlm}. Эти подходы, как правило, опираются на современные методы компьютерного зрения и глубокого обучения и демонстрируют высокую эффективность при работе с печатными текстами стандартного качества.

В то же время распознавание рукописных архивных текстов по-прежнему остаётся значительно менее изученной и более сложной задачей. Основными причинами этого являются высокая вариативность почерков, разнообразие начертаний иероглифов, а также наличие большого количества редких и древних символов, которые не используются в современных письменных источниках. Данные особенности существенно ограничивают применимость методов, ориентированных на современные печатные тексты.

Особый интерес в этом контексте представляют исторические документы, обладающие высокой культурной, исторической и гуманитарной ценностью. Однако для древних рукописей задача распознавания дополнительно осложняется низким качеством оцифрованных изображений, обусловленным износом носителей, повреждениями, неравномерной контрастностью и шумами сканирования \cite{shi2023m5hisdoc}. В связи с этим в последние годы было предложено множество подходов к распознаванию и поиску рукописных китайских иероглифов.

Существующие методы можно условно разделить на две основные группы. Первая группа включает традиционные методы обработки изображений, основанные на извлечении ручных признаков, геометрическом анализе и структурных описаниях \cite{zhu2016handwritten, wei2018compact, ma2021open}. Эти подходы, как правило, обладают высокой интерпретируемостью и не требуют больших объёмов обучающих данных. Вторая группа методов использует современные нейросетевые архитектуры, включая сверточные и трансформерные модели, которые демонстрируют высокое качество распознавания, но требуют значительных вычислительных ресурсов и больших размеченных наборов данных \cite{fujitake2024dtrocr, xu2022sophisticated, yang2017improving, zhong2016handwritten, zhuang2021handwritten, lin2021vibertgrid, coquenet2023faster}.

Для преодоления ограничений каждого из указанных направлений были предложены гибридные системы, сочетающие преимущества традиционных методов и глубокого обучения \cite{wang2014mqdf, liu2016offline, li2016handwritten}. Такие подходы позволяют повысить устойчивость распознавания и снизить требования к объёму обучающих данных, однако и они демонстрируют низкую эффективность в условиях крайне ограниченного числа примеров.

Данная проблема особенно ярко проявляется при работе с языками, обладающими большим алфавитом. Китайский язык является ярким примером такого случая: число иероглифов превышает 60 000, а с учётом устаревших и редко используемых символов может достигать более 100 000. Сбор репрезентативных обучающих наборов данных для всех возможных иероглифов в таких условиях становится практически невозможным.

Современные методы распознавания печатных китайских текстов, как правило, ориентированы на ограниченный набор часто используемых иероглифов, оцениваемый примерно в 10 000 знаков. Для этого подмножества существуют хорошо аннотированные датасеты, используемые для обучения нейросетевых моделей \cite{li2022large, sun2019chinese}. Однако при работе с историческими архивами такие данные оказываются недостаточными, поскольку значительная часть встречающихся в древних текстах иероглифов редко встречается или отсутствует в обучающих выборках. В результате обученные модели не способны корректно распознавать эти символы. В литературе предпринимались попытки формирования специализированных датасетов для древних иероглифов \cite{xu2019casia, ma2020joint, shi2023m5hisdoc}, однако результаты распознавания с использованием нейросетевых моделей для редко встречающихся символов остаются ограниченными \cite{shi2023m5hisdoc}, несмотря на существенные временные и трудовые затраты, связанные с созданием и разметкой подобных наборов данных.

Процесс ручной разметки исторических рукописей является крайне трудоёмким и дорогостоящим, требующим привлечения специалистов высокой квалификации. Более того, задачи автоматизации обработки древних архивов, являющихся частью национального культурного наследия, представляют интерес преимущественно для узкого круга исследователей — историков, филологов и лингвистов, и потому редко становятся объектом коммерческих разработок.

В связи с этим актуальной является задача поиска редких иероглифов в больших массивах сканированных рукописных документов в условиях отсутствия аннотированных данных для предварительного обучения. В обобщённой форме данная задача сводится к поиску наиболее похожих объектов среди большого числа кандидатов по заданному запросу. В литературе такой подход известен как однократное обучение (One-shot Learning) \cite{vinyals2016matching, snell2017prototypical, he2020memory}. Классическим примером данной парадигмы является задача распознавания лиц по одному изображению, широко применяемая в системах безопасности и идентификации личности. Методы однократного обучения также применялись для распознавания рукописных текстов \cite{koch2015siamese, gu2018meta, lake2015human}.

Ключевым элементом в задачах однократного обучения является выбор эффективного представления объектов и метрик для сравнения признаков. В этом контексте скелетные графы \cite{zhang1984fast, cychosz1994efficient, wang2018fully, livesu2012reconstructing} представляют собой перспективный инструмент для описания рукописных текстов и, в частности, китайских иероглифов, поскольку позволяют формализовать их топологическую и геометрическую структуру в виде вершин и рёбер. Для построения таких графов предложены различные методы, среди которых особый интерес представляют непрерывные модели скелетов \cite{siddiqi2008medial, LM2016continuous}, разработанные Л. М. Местецким. Эти модели обладают рядом преимуществ, включая математическую строгость, устойчивость к шумам и вычислительную эффективность.

В данной работе предлагается подход, основанный на прямом вычислении сходства иероглифов с использованием морфологического анализа изображений, методов вычислительной геометрии и линейного программирования, что позволяет эффективно решать задачу поиска редких иероглифов в условиях отсутствия размеченных обучающих данных.


\section{Основы математических средств}\label{sec:ch1/sec2}

TODO

\section{Выводы к главе \ref{ch:ch1}}

\FloatBarrier
