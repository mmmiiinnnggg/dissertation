\chapter{Обзор существующих методов и математических средств}\label{ch:ch1}

\section{Обзор методов распознавание иероглифов}\label{sec:ch1/sec1}
% Из PRIA
Распознавание текста документов всегда является одним из фокусов исследования для хранения, обработки и поиска информации. Для анализа печатных изображений документов в последние годы появились передовые технологии, которые решают задачу распознавания с достаточным качеством \cite{coquenet2022end} и уже применяются в разных системах для автоматизации рутинных процессов. Однако рукописные архивные тексты остаются малоизученной областью из-за вариативности и разнообразия начертания иероглифов, наличия в текстах редких древних символов, не используемых в современных текстах.

Исторические документы привлекают интерес гуманитарных исследователей за счет их большой культурной и человеческой ценности, но для древних рукописей задача распознавания дополнительно усложняется низким качеством отсканированных изображений документов \cite{shi2023m5hisdoc}. Разрабатываются различные методы для решения задачи распознавания и поиска рукописных китайских иероглифов, включая и традиционные методы обработки изображений, и современные методы с помощью нейронных сетей \cite{xu2022sophisticated}.

В борьбе с этими трудностями в последнее время развивались методы, использующие глубокое обучение, требующие составления больших наборов обучающих данных \cite{fujitake2024dtrocr}, а также мощных вычислительных ресурсов. Однако такие методы плохо обучаются в случае малого числа примеров изображений, входящих в набор обучающих данных. Для целевых языков с большим алфавитом, таких как китайский, у которого алфавит содержит более 60000 иероглифов, а в некоторых справочниках упоминается даже цифра более 100000, включая устаревшие и редко используемые, становится крайне трудно собирать наборы обучающих данных для применения метода машинного обучения.

Методы распознавания китайских печатных текстов ориентированы на работу с современным набором иероглифов, который оценивается в 10 тыс. знаков \cite{li2022large}. Для такого набора существуют аннотированные датасеты, используемые для обучения нейросетевых алгоритмов распознавания \cite{xu2019casia,ma2020joint}. Но при работе с историческими архивами использование таких данных оказывается недостаточным. В древних текстах встречается очень много иероглифов, вышедших из употребления, которых нет в размеченных текстах. Они не участвуют в обучении, поэтому обученные алгоритмы не способны их распознавать. При этом разметка большого количества древних текстов является очень трудоёмким и дорогостоящим делом, требующим привлечения высококвалифицированных специалистов. А задачи автоматизации работы с древними архивами, составляющими национальное культурное наследие, представляют интерес лишь для небольшого сообщества профессиональных исследователей – филологов, историков, лингвистов. Они не привлекают внимания разработчиков коммерческого программного обеспечения.

Таким образом, актуальной задачей поиска редких иероглифов в огромных файлах сканированных рукописных текстов является разработка методов, способных осуществлять поиск при отсутствии аннотированных данных для предварительного обучения. В данной работе мы предлагаем подход, состоящий в прямом вычислении сходства иероглифов на основе морфологического анализа изображений иероглифов с применением методов вычислительной геометрии и линейного программирования.


\section{Основы математических средств}\label{sec:ch1/sec2}

TODO

\section{Выводы к главе \cref{ch:ch1}}
\FloatBarrier
