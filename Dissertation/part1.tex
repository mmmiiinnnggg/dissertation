\chapter{Обзор методов распознавания иероглифических текстов при расшифровке и поиске}\label{ch:ch1}

В данной главе рассматривается широкий круг методов распознавания иероглифических текстов, которые на протяжении нескольких десятилетий остаются одной из наиболее сложных и актуальных задач в области компьютерного зрения, анализа изображений и обработки документов.

Исторически исследования в данной области развивались в двух взаимосвязанных, но концептуально различных направлениях. Первое направление ориентировано на \textit{задачу распознавания и декодирования иероглифов}, то есть на установление соответствия между изображением символа и его семантическим или кодовым представлением. В рамках этого подхода основной целью является получение последовательности символов в машиночитаемом виде, что позволяет использовать распознанный текст для дальнейшей лингвистической обработки, индексирования, перевода и анализа содержания. Второе направление связано с \textit{задачами поиска}, сопоставления и группировки иероглифических форм, где ключевым является не столько точное восстановление символа, сколько выявление графического сходства между изображениями, независимо от их интерпретации на уровне языка.

В отличие от алфавитных систем письма, особая сложность распознавания иероглифических текстов, таких как китайская, обусловлена рядом факторов. Во-первых, количество различных иероглифов может достигать десятков тысяч, при этом многие из них встречаются крайне редко, что приводит к проблеме дефицита обучающих данных. Во-вторых, иероглифы характеризуются сложной внутренней структурой, включающей радикалы, штрихи и их пространственные отношения, которые могут изменяться в зависимости от стиля письма, эпохи, носителя или технических условий оцифровки. В-третьих, значительную роль играют деградации изображения, такие как шум, разрывы штрихов, искажения, а также особенности исторических документов, включая нестандартизированные формы и архаические варианты написания.

В рамках данной главы проводится систематический обзор существующих методов, применяемых для решения указанных задач. Рассматриваются как традиционные подходы, основанные на ручном проектировании признаков, структурном анализе и сопоставлении графов или скелетных представлений, так и современные методы машинного обучения и глубоких нейронных сетей. Особое внимание уделяется принципиальным различиям между методами, ориентированными на распознавание символов как конечных классов, и подходами, направленными на извлечение инвариантных графических представлений, пригодных для задач поиска и сопоставления. Кроме того, в обзоре анализируются достоинства и ограничения различных классов методов с точки зрения их применимости к различным сценариям, включая распознавание печатных и рукописных текстов, работу с историческими архивами, а также задачи с ограниченным числом размеченных данных, такие как однократное (one-shot) и малократное (few-shot) обучение.

Таким образом, данная глава формирует теоретическую и методологическую основу для последующего анализа и разработки методов распознавания иероглифических текстов, а также позволяет наглядно продемонстрировать эволюцию подходов от классических алгоритмов к современным нейросетевым решениям в контексте задач расшифровки и поиска.

\section{Методы, основанные на принципе расшифровки}\label{sec:ch1/sec1}
Методы, относящиеся к принципу расшифровки, являются на сегодняшний день основным и наиболее широко распространённым направлением в области распознавания текстов, включая распознавание иероглифических письменностей. Данный класс подходов ориентирован на непосредственное восстановление символической последовательности по входному изображению, то есть на установление однозначного соответствия между графическим образом иероглифа и его кодовым или лексическим представлением. Именно такие методы лежат в основе большинства современных систем оптического распознавания текста (OCR), применяемых для обработки печатных и рукописных документов, архивных материалов и цифровых библиотек.

В рамках данной группы методов условно можно выделить три основных класса. Первый класс составляют традиционные методы, основанные на признаковых дескрипторах, в которых ключевую роль играет ручное извлечение структурных, геометрических и топологических характеристик иероглифов. Второй класс включает статистические методы расшифровки, использующие вероятностные модели, методы классификации и последовательностного декодирования. Наконец, третий и наиболее активно развивающийся класс представлен нейросетевыми методами, основанными на глубинном обучении.

\subsection{Традиционные методы признаковых дискрипторов}
\subsection{Статистические методы расшифровки иероглифов}
\subsection{Нейросетевые методы расшифровки иероглифов}
\section{Методы, основанные на принципе поиска}\label{sec:ch1/sec2}
Методы, применяемые при принципе поиска, представляют собой альтернативное и в ряде прикладных сценариев принципиально важное направление в области анализа иероглифических текстов. В отличие от методов расшифровки, ориентированных на восстановление точного символического представления, поисковые подходы нацелены прежде всего на выявление, сопоставление и группировку графических форм по степени их визуального или структурного сходства. Такие методы особенно востребованы в задачах анализа исторических документов, архивных коллекций и малоизученных письменных источников, где полная расшифровка текста затруднена или невозможна из-за отсутствия словарей, редкости символов или сильной деградации изображений.

Ключевой особенностью поисковых методов является их относительная независимость от полного набора классов иероглифов и строгой лингвистической интерпретации. Вместо этого они оперируют понятием графической близости, что позволяет решать задачи поиска в больших массивах изображений, обнаружения повторяющихся символов, сопоставления вариантов написания и поддержки экспертной расшифровки.

В рамках данной группы методов можно выделить три основных класса подходов. Первый класс составляют методы шаблонного сопоставления, в которых сравнение иероглифов осуществляется напрямую на уровне изображений или их структурных представлений, таких как контуры, скелеты или графы штрихов. Второй класс включает метрические методы в признаковом пространстве, где иероглифы отображаются в векторное пространство признаков, а задача поиска формулируется как вычисление расстояний или мер сходства между представлениями. Третий класс представлен методами, основанными на предобученных представлениях и инженерии подсказок (prompt engineering), которые заимствуют идеи из современных моделей глубокого обучения и больших языковых и мультимодальных моделей.

\subsection{Методы шаблонного сопоставления}
\subsection{Метрические методы в признаковом пространстве}
\subsection{Предобученные представления и методы инженерии подсказок (Prompt Engineering)}

TODO: Вносить каждые части перед главами.
% \section{Основы математических средств}\label{sec:ch1/sec2}

% \subsection{Непрерывная морфология и медиальное представление}\label{sec:ch1/sec2/sub1}

% \subsection{Геометрический и топологический граф}\label{sec:ch1/sec2/sub2}

% \subsection{Задачи о назначений и линейное программирование}
% \label{sec:ch1/sec2/sub3}

\section{Выводы к главе \ref{ch:ch1}}

TODO: текущие подходы обладают определеннымм недостотками, что и вытекает резонность предложенного метода.

\FloatBarrier
