\chapter{Практические приложения контекстного поиска}\label{ch:ch5}
Разработанные в предыдущих главах методы построения точечного признакового описания на основе непрерывных морфологических моделей и оптимизационные модели вычисления меры сходства через задачу о назначениях легли в основу программного комплекса, предназначенного для анализа и поиска иероглифических текстов. Предлагаемый метод был реализован и прошел экспериментальную апробацию на реальных данных отсканированных древнекитайских документов. Проведенные вычислительные эксперименты направлены на комплексную оценку работоспособности и эффективности предложенного подхода при решении задач поиска иероглифов в массивах исторических архивов.

Следует отметить, что требования, предъявляемые к алгоритмам распознавания для поиска иероглифов, отличаются от требований к алгоритмам расшифровки текста, которые решают задачу перевода его в машиночитаемый формат. Алгоритмы расшифровки должны обеспечить баланс ошибок ложного узнавания (FAR – False Acceptance Rate) и пропуска цели (FRR – False Rejection Rate). Для алгоритмов поиска редких иероглифов эти критерии не равнозначны. Уровень ошибок FAR допускается существенно более высокий, чем FRR. Это объясняется тем, что найденные в результате поиска иероглифы должны пройти визуальный контроль эксперта асессора, который легко может быть реализован в диалоговом режиме. В процессе такого контроля эксперт может отвергнуть ошибочно найденные иероглифы-кандидаты. С этой работой он легко справится даже если среди найденных алгоритмом кандидатов доля правильных составит 5-10\%. В то же время пропуск цели при решении задачи поиска может оказаться фатальным, поскольку пропущенный иероглиф в большом файле будет потерян.

Таким образом, успех работы алгоритма определяется вероятностью попадания искомых символов в верхнюю часть (топ) ранжированного списка, сформированного по убыванию меры сходства с запросом. В соответствии с этими положениями, в качестве основного критерия эффективности в данной работе используются оценки вероятности обнаружения иероглифа в зависимости от размера выбранной «топовой» части списка кандидатов.

\section{Методы ускорения алгоритма}\label{sec:ch5/sect0}
\subsection{Распараллеливание сравнений изображений}\label{subsec:ch5/sect0/sub1}

TODO: Распараллеливание сравнений изображений

\subsection{Предварительный подсчет признаков изображений}\label{subsec:ch5/sect0/sub2}

TODO: Предварительный подсчет признаков изображений

\section{Задача однократного распознавания}\label{sec:ch5/sect1}
Чтобы сравнение предлагаемого метода с существующими подходами было справедливым и масштабированным, мы провели эксперименты однократного распознавания на крупном аннотированном корпусе древнекитайских рукописных символов CASIA-AHCDB \cite{xu2019casia}. Этот корпус представляет собой одну из наиболее объёмных и тщательно размеченных коллекций древних рукописных иероглифов: он включает более 2.2 миллиона примеров из более чем 10~000 классов, что делает его особенно ценным для задач масштабного поиска и сопоставления символов в реальных документах. База данных CASIA-AHCDB была создана путём ручной разметки большого массива страниц древнекитайских рукописей и предоставляет аннотированные изображения отдельных иероглифов, извлечённых из исторических источников. Общее время для аннотирования оценивается в несколько тысячи часов. Благодаря широкому разнообразию почерков, стилевых вариаций и шумов, характерных для древних текстов, этот набор данных хорошо подходит для проверки устойчивости алгоритмов поиска и распознавания вне ограничений стандартных современных наборов. Использование такого набора в качестве тестовой базы обеспечивает строгую проверку метода в условиях, приближённых к практическим задачам историко-лингвистического анализа, где высокая вариативность форм и большое количество классов существенно усложняют задачу поиска.

Задача однократного распознавания состоит в нахождении самого похожего иероглифа среди набора примеров-кандидатов с определенным количеством (5-20 штук) при заданном иероглифе запроса. В набор кандидатов входит один положительный пример, который является другим изображением иероглифа запроса, и несколько отрицательных примеров, на которых изображены другие иероглифы. Если самый похожий иероглиф, выданный алгоритмом, является положительным примером, то считается, что для данной задачи эксперимент дал положительный выход. Согласно \cite{liu2022one}, мы проводили по 550 экспериментов с рандомно взятыми изображениями иероглифов из большого размеченного набора данных и посчитали итоговую точность классификации по следующей формуле:
\[
    \text{Accuracy} = \frac{\text{Number of experiments with positive outcome}}{\text{Total number of experiments}}
\]

Все эксперименты проводились на обычном ноутбуке с восьмиядерным процессором Intel\textsuperscript{\textregistered} Core\textsuperscript{\texttrademark} i5-11300H 11-го поколения с тактовой частотой 3,10 ГГц и 16 ГБ оперативной памяти. Данная конфигурация выгодно сопоставима с современными методами на основе нейронных сетей, использующими несколько графических ускорителей (GPU).

Результат экспериментов представлен в таблице \cref{tab:result_1}. Мы выбрали работы \cite{liu2022one,li2019one} в качестве базовых методов для сравнения, поскольку это последние передовые методы, использующие нейронные сети, и мы проводили эксперименты в той же самой конфигурации, что и в этих работах. Для каждого эксперимента мы использовали 5 и 20 изображений для формирования набора кандидатов в соответствии с \cite{liu2022one}, которые в таблице \cref{tab:result_1} обозначены «5-way» и «20-way». Стоит отметить, что несмотря на то, что предложенный метод реализует алгоритмический подход, его точность классификации сопоставима или даже превышает точность последних передовых методов, использующих нейронные сети и мощные вычислительные устройства. Это свидетельствует об эффективности нашего метода.

TODO: 1. Добавить random guess, чтобы показать нетривиальное решение. 2. Добавить примеры неудачи и проанализировать почему так произошло.

\begin{table}
    \centering
    \captionsetup{justification=centering} % выравнивание подписи по-центру
    \caption{Сравнение точности классификации методов однократного распознавания}\label{tab:result_1}
    \begin{tabular}{cccc}
        \toprule
        \textbf{Метод}                   & \textbf{Категория}       & \textbf{5-way} & \textbf{20-way} \\
        \midrule
        Сиамская сеть \cite{li2022large} & Нейросетевой             & 95.00          & -               \\
        MED \cite{ma2020joint}           & Нейросетевой             & 96.64          & 94.52           \\
        \textbf{Предлагаемый метод}      & \textbf{Алгоритмический} & \textbf{96.91} & \textbf{95.10}  \\
        \bottomrule
    \end{tabular}
\end{table}

\section{Поисковый запрос в рукописных документах}\label{sec:ch5/sect2}

Для оценки эффективности предлагаемого метода в условиях, приближённых к реальному сценарию поиска иероглифов в рукописных документах, в работе были использованы отсканированные страницы из набора данных MTHv2 \cite{ma2020joint}. Данный набор данных содержит около 3500 изображений документов различных форматов и относится к числу наиболее репрезентативных корпусов рукописных китайских материалов, доступных в открытом доступе.

Каждое изображение в наборе MTHv2 снабжено тремя типами аннотаций. Первый тип включает аннотации на уровне строк, содержащие информацию о положении всех текстовых строк на странице, а также соответствующее текстовое содержимое. Второй тип представлен аннотациями на уровне отдельных символов и содержит координаты ограничивающих рамок для каждого китайского иероглифа. Третий тип аннотаций описывает граничные области документа и задаёт координаты различных структурных элементов страницы, включая иллюстрации и текстовые блоки.

Поскольку в рамках настоящей работы основное внимание уделяется анализу и сравнению отдельных иероглифических изображений, в экспериментах используются исключительно аннотации на уровне символов. Такой выбор позволяет сосредоточиться на задаче сегментации и сопоставления отдельных экземпляров, не привлекая дополнительную информацию о строковой или макетной структуре документа.

Примеры страниц из набора данных MTHv2 приведены на рисунках~\cref{fig:real_manuscript_1,fig:real_manuscript_2}. Как видно из представленных примеров, документы характеризуются наличием помех и шумов различной природы, обусловленных длительным хранением рукописей в архивах, физическим износом носителей, а также нестабильным качеством оцифровки. Эти факторы приводят к разрывам штрихов, неравномерности контуров и фоновым искажениям, что делает данный набор данных особенно сложным и показательным для оценки устойчивости предлагаемых методов в практических условиях.

Задача поиска формулируется как нахождение всех вхождений заданного иероглифического запроса в представленных рукописных документах. Результатом работы алгоритма является упорядоченный список изображений иероглифов, ранжированный по величине меры сходства между запросом и каждым кандидатом из документа. Качество поиска оценивается по доле истинных вхождений целевого иероглифа, попавших в верхнюю часть ранжированного списка (топ-список). Чем выше данный показатель, тем меньше вероятность ошибки пропуска цели, то есть ситуации, при которой иероглифы, совпадающие с запросом, не обнаруживаются или располагаются на низких позициях в результатах поиска. Для проведения экспериментов были отобраны четыре документа из набора данных, в которых были выбраны иероглифы, число экземпляров изображений которых превышает 20. Такой отбор обеспечивает статистически значимую оценку качества поиска для каждого запроса. В общей сложности экспериментальный набор включает около 1600 изображений иероглифов.
\begin{figure}[ht]
    \centerfloat{
        \includegraphics[width=0.8\linewidth]{real_manuscript_1.jpg}
    }
    \caption[Пример страниц набора документов MTHv2, первый документ]{Пример страниц набора документов MTHv2, первый документ.}\label{fig:real_manuscript_1}
\end{figure}
\begin{figure}[ht]
    \centerfloat{
        \includegraphics[width=0.8\linewidth]{real_manuscript_2.jpg}
    }
    \caption[Пример страниц набора документов MTHv2, второй документ]{Пример страниц набора документов MTHv2, второй документ.}\label{fig:real_manuscript_2}
\end{figure}

Для каждого выбранного иероглифа был выполнен поиск по всему множеству из 1600 изображений, при этом в качестве запроса использовался один произвольно выбранный экземпляр данного иероглифа. Итоговая оценка качества поиска получена путём усреднения показателей доли истинных вхождений по всем иероглифам для различных уровней отсечения ранжированного списка — от топ-5\% до топ-50\%. Все эксперименты проводились на том же ноутбуке, использованный в предыдущей подглаве.

Результаты экспериментального исследования представлены в таблице~\cref{tab:result_2}. Как следует из приведённых данных, при отборе первых 10\% иероглифов в ранжированном списке предлагаемый метод обеспечивает локализацию истинных вхождений с точностью, превышающей 90\%. Это означает, что при практическом использовании разработанного подхода конечные пользователи могут существенно сократить объём документов, подлежащих ручному анализу, по меньшей мере в десять раз, не сталкиваясь при этом с заметными потерями в полноте поиска.

В целом полученные результаты подтверждают высокую эффективность предлагаемого метода в условиях реальных рукописных архивов и демонстрируют его практическую применимость для задач навигации и поиска иероглифов при ограниченных вычислительных и временных ресурсах.
\begin{table}
    \centering
    \captionsetup{justification=centering} % выравнивание подписи по-центру
    \caption{Результат поиска иероглифов при разных долях упорядоченного списка вычисленных мер сходства между запросом и изображениями из файла}\label{tab:result_2}
    \begin{tabular}{ccccc}
        \toprule
        \textbf{Доля}     & \textbf{Top5\%} & \textbf{Top10\%} & \textbf{Top20\%} & \textbf{Top50\%} \\
        \midrule
        Процент вхождения & 81.25           & 90.50            & 95.50            & 98.75            \\
        \bottomrule
    \end{tabular}
\end{table}


\section{Выводы к главе \cref{ch:ch5}}
\clearpage