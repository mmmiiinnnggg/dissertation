\chapter{Методы построения точечного признакового описания иероглифических текстов}\label{ch:ch2}

Построение геометрической модели иероглифа основано на методах вычислительной геометрии, что обеспечивает математическую корректность модели, устойчивость вычислительных процедур и высокую вычислительную эффективность алгоритмов сравнения и поиска иероглифов.

\section{Генерация признаков по аналоги созвездий}\label{sec:ch2/sec1}

\section{Построение медианной оси и скелетного графа}\label{sec:ch2/sec2}
Предложенное решение основано на построении морфологической модели иероглифа в виде геометрического графа. Этот граф называется метаграфом иероглифа. Метаграф строится на основе преобразования исходного цифрового изображения иероглифа.

Построение метаграфа иероглифа основано на использовании медиального представления формы изображений \cite{siddiqi2008medial}. Медиальное представление состоит из двух компонентов: медиальной оси и радиальной функции. Медиальная ось фигуры, также называемая скелетом – это множество центров всех вписанных в фигуру окружностей. Радиальная функция задается в каждой точке скелета и равна радиусу максимальной пустой окружности с центром в этой точке. Преимущество такой модели заключается, прежде всего, в ее строгом математическом определении, высокой точности, а также в вычислительной эффективности алгоритмов.

\section{Получение описания иероглифа конечным набором точек}\label{sec:ch2/sec3}


\subsection{Выбор критичных точек с максимальной кривизной}\label{sec:ch2/sec3/sub1}

\subsection{Метод нормализации}\label{sec:ch2/sec3/sub2}

\subsection{Методы борьбы с аномальными точками: смерживание близких по расстоянию вершин для снижения размерности задачи (merging\_nodes), удаление мелких связанных компонент}\label{sec:ch2/sec3/sub3}

\FloatBarrier
