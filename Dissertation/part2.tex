\chapter{Методы построения точечного признакового описания иероглифических текстов}\label{ch:ch2}
Построение геометрической модели иероглифа основано на методах вычислительной геометрии, что обеспечивает математическую корректность модели, устойчивость вычислительных процедур и высокую вычислительную эффективность алгоритмов сравнения и поиска иероглифов.

\section{Генерация признаков по аналоги созвездий}\label{sec:ch2/sec1}
% Из ММРО/PRIA

Предлагаемый подход состоит в построении информативного признакового описания структуры иероглифа в виде небольшого по размеру множества точек, называемого созвездием иероглифа. Созвездие формируется на основе срединной оси иероглифа, также называемой скелетом, в виде планарного геометрического графа. Для дальнейшего определения сходства и различия иероглифов созвездия используются в качестве признаков.

Скелеты традиционно применяются в методах распознавания формы объектов на изображениях \cite{siddiqi2008medial}, поскольку они хорошо описывают топологическую и геометрическую структуру объектов. В задачах распознавания рукописных текстов скелет интерпретируется как след пера – траектория движения пишущего инструмента при написании текста. Для изображений иероглифов скелет представляет собой геометрический граф сложной структуры. Особенности почерка автора, а также шумы, вносимые дефектами хранения и сканирования рукописей, приводят к тому, что скелеты изображений одного и того же иероглифа не являются изоморфными графами, и имеют различия в геометрии линий рёбер. Предлагаемый подход к сравнению на основе созвездий позволяет сравнивать такие графы и определять сходство и различие любых двух изображений иероглифов: как одного и того же, так и разных.

Методы сравнения формы объектов, основанные на анализе дискретных множеств точек, находят применение при решении прикладных задач. Одним из примеров подобного подхода может служить задача идентификации созвездий на звёздном небе. Созвездие представляет собой совокупность звёзд, визуально образующих устойчивую конфигурацию на небесной сфере. Современные астрономические каталоги выделяют 88 созвездий, каждое из которых характеризуется уникальной структурой, определяемой взаимным расположением входящих в него звёзд. В состав каждого созвездия входит от 10 до 140 звёзд с видимой яркостью до 6-й звёздной величины. Процесс распознавания созвездий основывается на выделении небольшого подмножества звёзд, формирующих наиболее узнаваемый фрагмент конфигурации. Так, например, на рисунке \cref{fig:constellation_1} созвездие Большая Медведица, включающее около 126 звёзд, различимых невооружённым глазом, легко идентифицируется по семи наиболее ярким звёздам, образующим фигуру Большого Ковша — так называемый астеризм, традиционно выделяемый как самостоятельная группа. Термин «метод созвездий» в настоящей работе используется применительно к сравнению иероглифов, поскольку обе задачи демонстрируют очевидное сходство в структурной организации и принципах распознавания.

\begin{figure}[ht]
    \centerfloat{
        \includegraphics[width=0.8\linewidth]{constellation_1.jpg}
    }
    \caption[Пример модели для сравнения формы объектов на основе конечных малых наборов их точек: созвездие Большая Медведица.]{Пример модели для сравнения формы объектов на основе конечных малых наборов их точек: созвездие Большая Медведица.}\label{fig:constellation_1}
\end{figure}

Другая хорошо известная модель сравнения формы объектов по конечным небольшим наборам их точек используется в дактилоскопии. Пример такой модели показан на рисунке \cref{fig:constellation_2}. Для идентификации личности на изображении отпечатка пальцев выделяются наборы точек, называемые минуциями. Набор минуций является признаковым описанием изображения отпечатка пальца. Сравнение отпечатков выполняется с помощью специальной меры, которая вычисляется на основе наложения карт минуций и вычисления парных расстояний между ними.

\begin{figure}[ht]
    \centerfloat{
        \includegraphics[width=0.4\linewidth]{constellation_2.jpg}
    }
    \caption[Пример модели для сравнения формы объектов на основе конечных малых наборов их точек: отпечатка пальцев и минуций]{Пример модели для сравнения формы объектов на основе конечных малых наборов их точек: отпечатка пальцев и минуций.}\label{fig:constellation_2}
\end{figure}

\section{Построение медианной оси и скелетного графа}\label{sec:ch2/sec2}
% Из ISPRS
Предложенное решение основано на построении морфологической модели иероглифа в виде геометрического графа. Этот граф называется метаграфом иероглифа. Метаграф строится на основе преобразования исходного цифрового изображения иероглифа.

Построение метаграфа иероглифа основано на использовании медиального представления формы изображений \cite{siddiqi2008medial}. Медиальное представление состоит из двух компонентов: медиальной оси и радиальной функции. Медиальная ось фигуры, также называемая скелетом – это множество центров всех вписанных в фигуру окружностей. Радиальная функция задается в каждой точке скелета и равна радиусу максимальной пустой окружности с центром в этой точке. Преимущество такой модели заключается, прежде всего, в ее строгом математическом определении, высокой точности, а также в вычислительной эффективности алгоритмов.

Общая последовательность действий при построении метаграфа представлена на рисунке \cref{fig:metagraph}. Она включает в себя следующие шаги:
\begin{enumerate}
    \item Бинаризация исходного изображения иероглифа. Это процесс преобразования исходного изображения в черное-белое растровое изображение, которое является базой для построение скелетного графа с помощью морфологических операций.
    \item Аппроксимация бинарного изображения иероглифа многоугольной фигурой. Выбор многоугольных фигур для аппроксимации формы иероглифов обусловлен возможностью использования эффективных алгоритмов вычислительной геометрии для построения их скелетов на основе обобщенных диаграмм Вороного \cite{siddiqi2008medial}.
    \item Построение непрерывного скелета многоугольной фигуры. Скелет — это множество точек центров максимальных вписанных в фигуру окружностей. Получение непрерывного скелета осуществляется посредством построения обобщенной диаграммы Вороного многоугольной фигуры и последующей фильтрации рёбер диаграммы Вороного.
    \item Регуляризация (стрижка) скелета, состоящая в последовательном отсечении части ребер скелета. Этот процесс называется стрижкой (pruning). Использование предлагаемой непрерывной модели позволяет получить математически строгий критерий стрижки, основанный на концепции силуэта скелетного подграфа \cite{LM2016continuous}.

\end{enumerate}

\begin{figure}[ht]
    \centerfloat{
        \hfill
        \subcaptionbox{Первый подрисунок\label{fig:metagraph_a}}{%
            \includegraphics[width=0.3\linewidth]{metagraph_a.jpg}}
        \hfill
        \subcaptionbox{Второй\label{fig:metagraph_b}}{%
            \includegraphics[width=0.3\linewidth]{metagraph_b.jpg}}
        \hfill
        \subcaptionbox{Второй\label{fig:metagraph_c}}{%
            \includegraphics[width=0.3\linewidth]{metagraph_c.jpg}}
        \hfill
        \par\bigskip
        \hfill
        \subcaptionbox{Второй\label{fig:metagraph_d}}{%
            \includegraphics[width=0.3\linewidth]{metagraph_d.jpg}}
        \hfill
        \subcaptionbox{Второй\label{fig:metagraph_e}}{%
            \includegraphics[width=0.3\linewidth]{metagraph_e.jpg}}
        \hfill
        \subcaptionbox{Второй\label{fig:metagraph_f}}{%
            \includegraphics[width=0.3\linewidth]{metagraph_f.jpg}}
        \par\bigskip
        \hfill
        \subcaptionbox{Второй\label{fig:metagraph_g}}{%
            \includegraphics[width=0.3\linewidth]{metagraph_g.jpg}}
        \hfill
        \subcaptionbox{Второй\label{fig:metagraph_h}}{%
            \includegraphics[width=0.3\linewidth]{metagraph_h.jpg}}
        \hfill
        \subcaptionbox{Второй\label{fig:metagraph_i}}{%
            \includegraphics[width=0.3\linewidth]{metagraph_i.jpg}}
        \hfill
    }
    \caption[Пример построения метаграфа иероглифа для цифрового изображения]{Пример построения метаграфа иероглифа для цифрового изображения.}\label{fig:metagraph}
\end{figure}

Выбор многоугольных фигур для аппроксимации формы иероглифов обусловлен возможностью использования эффективных алгоритмов вычислительной геометрии для построения их медиального представления на основе диаграмм Вороного.

Аппроксимирующая многоугольная фигура может состоять из нескольких компонент связности, каждая из которых представляет собой многоугольник с многоугольными отверстиями (рисунок \cref{fig:metagraph_e}). Аппроксимация иероглифа многоугольной фигурой показана на рисунках \cref{fig:metagraph_a}-\cref{fig:metagraph_d}. Для исходного растрового изображения иероглифа (рисунок \cref{fig:metagraph_b}) необходимо выполнить поиск, трассировку и аппроксимацию всех граничных контуров фигуры \cite{mestetskiy2008binary}.

Построенные многоугольники для всех граничных контуров образуют аппроксимирующую многоугольную фигуру для изображения иероглифа (рисунок \cref{fig:metagraph_e}).

Следующим шагом предлагаемого подхода является построение внутренней диаграммы Вороного (ДВ) полученной многоугольной фигуры (рисунок \cref{fig:metagraph_f}). Для этого граничные многоугольники фигуры разбиваются на подмножества, называемые сайтами. Сайтами являются все вершины и все стороны граничных многоугольников. Внутренняя ДВ многоугольной фигуры представляет собой разбиение множества точек фигуры на локусы – подмножества точек, ближайших к одному из сайтов фигуры. Для построения внутренней ДВ мы используем алгоритм \cite{mestetskiy2024constructing}.

Локусы ДВ пересекаются в точках их границ. На рисунке \cref{fig:metagraph_f} эти пересечения показаны красными линиями. Этот набор линий называется ребрами ДВ. Его можно рассматривать как плоский геометрический граф. Скелет – это подграф ДВ, в него входят почти все ребра ДВ, кроме ребер, инцидентных локусам вогнутых вершин многоугольной фигуры. Таким образом, скелет можно получить на основе отсечения этих ребер ДВ (рисунок \cref{fig:metagraph_f}).

Далее строится метаграф иероглифа, представляющий собой подграф скелета. Процесс выделения подграфа сводится к последовательному отсечению части ребер скелета. Этот процесс называется стрижкой. Использование предлагаемой непрерывной модели позволяет получить математически строгий критерий стрижки, основанный на концепции силуэта скелетного подграфа \cite{mestetskiy2024constructing}. Силуэтом скелетного подграфа называется объединение всех вписанных кругов фигуры с центрами на вершинах и рёбрах подграфа. Общая идея стрижки состоит в том, чтобы выделить в скелете минимальный подграф, у которого силуэт отличается от многоугольной фигуры не более чем на заданную пороговую величину в метрике Хаусдорфа. Необходимо последовательно отсекать те терминальные ребра подграфа, при удалении которых расстояние Хаусдорфа между силуэтом подграфа и многоугольной фигурой не превышает заданного порога. В качестве порогового значения для изображений иероглифов может быть выбран размер пикселя. Пример полученного в результате стрижки подграфа показан на рисунках \cref{fig:metagraph_g}-\cref{fig:metagraph_h}.

\section{Получение описания иероглифа конечным набором точек}\label{sec:ch2/sec3}
Формирование точечных признаков представляет собой включение характерных точек полученного метаграфа, сгенерированного с помощью математической морфологии. В подграфе скелета, полученном в результате стрижки из предыдущей подглавы, имеются вершины 1, 2 и 3 степени. Нетривиальный выбор является включением терминальных и разветвленных вершин, то есть вершин 1 и 3 степени. На практике оказывается, что такой выбор недостачен при получения адекватного описания иероглифа.

TODO: Добавить пример изображений иероглифов того, что только 1,3 степени не достаточно

\subsection{Выбор критичных точек с максимальной кривизной}\label{sec:ch2/sec3/sub1}
Множество вершин метаграфа формируется на основе двух правил:
\begin{enumerate}
    \item Вершинами метаграфа считаются все вершины 1 и 3 степени этого подграфа, то есть концевые и разветвлённые узлы скелетных ветвей.
    \item Некоторое подмножество вершин 2 степени также объявляется вершинами метаграфа для учёта данных о форме рёбер метаграфа. Это вершины максимальной кривизны в цепях рёбер, соединяющих вершины 1 и 3 степени.
\end{enumerate}
Оценка кривизны ветви скелета может быть сделана на основе анализа углов ломаной линии, описывающей ветвь скелета как цепь графа. Если угол между соседними сторонами фигуры меньше заданного порога, то вершина представляет собой точку максимальной кривизны. Рёбрами метаграфа являются все цепи скелета, соединяющие отобранные вершины. Форма цепи определяется положением промежуточных вершин 2 степени.

\subsection{Метод нормализации}\label{sec:ch2/sec3/sub2}
% Из PRIA
Задача определения сходства и различия иероглифов основана на идее сравнения положения точек, входящих в созвездия иероглифов. Для корректного сравнения созвездия должны быть нормализованы, т.е. преобразованы в определённый стандартный формат, что позволяет сравнивать их между собой. Нормализация созвездий включает два преобразования: центрирование и масштабирование. Центрирование преобразует созвездие к единой системе координат, а масштабирование приводит созвездия к единому масштабу. Нормализация основывается на определении понятия центра и отклонения для фигуры иероглифа. При описании иероглифа в виде созвездия существует простой приемлемый по точности способ, состоящий в распределении некоторой «массы» по точкам созвездия, причем масса всех точек одинакова.

Центрирование основано на вычислении центра массы созвездия. Центр массы (или геометрический центр) вычисляется как среднее значение координат всех вершин, входящих в точечное представление иероглифа. Пусть $P=\{p_i=(x_i,y_i)|i=0,1,…,N\}$ – координаты точек созвездия, тогда центр массы $c=(x_c, y_c)$ вычисляется следующим образом:
\[
    c=(x_c, y_c)=\frac{1}{N} \sum_{i=1}^{N}p_i = (\sum_{i=1}^{N}x_i,\sum_{i=1}^{N}y_i)\]
Далее масштабирование осуществляется относительно полученного центра массы. Преобразование состоит в сдвиге и умножении координат точек созвездий на коэффициент, определяемый на основе вычисления среднеквадратичного отклонения $\sigma$ точек созвездия от центра массы
\[
    \sigma = \sqrt{\frac{1}{N}\sum_{i=1}^{N}\|p_i-c\|^2} = \sqrt{\frac{1}{N}\sum_{i=1}^{N}(x_i-x_c)^2+(y_i-y_c)^2}
\]
Координаты точек преобразуются таким образом, чтобы созвездия имели примерно одинаковую пространственную протяженность независимо от исходного размера, т.е. отклонение равнялось единице. При масштабировании созвездия координаты пересчитываются по формуле:
\[
    \tilde{p_i} = \frac{p_i-c}{\sigma} = (\frac{x_i-x_c}{\sigma},\frac{y_i-y_c}{\sigma})
\]


\subsection{Методы борьбы с аномальными точками: смерживание близких по расстоянию вершин для снижения размерности задачи (merging\_nodes), удаление мелких связанных компонент}\label{sec:ch2/sec3/sub3}

TODO:

\section{Выводы к главе \cref{ch:ch2}}

Объединение топологических и геометрических параметров позволяет сформировать информативное множество признаков, устойчивых к небольшим деформациям и вариациям почерка. В дальнейшем полученные точечные дескрипторы используются при сопоставлении и поиске иероглифов, обеспечивая возможность сравнения их структур по аналогии с методом «созвездий».

\FloatBarrier
