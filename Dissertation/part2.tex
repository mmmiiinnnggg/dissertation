\chapter{Навигация в цифровых древнекитайских рукописных архивах}\label{ch:ch2}
\section{Особенности древнекитайских рукописных текстов}\label{sec:ch2/sec1}
% 1. невозможность сбора полностью всех иероглифов, включающих редкие встречающие.
% 2. некоммерческая задача, за которую интересует только узкий кург профиссианалов. культурное наследие - миссия университетов.
% 3. Критерий ошибки в задаче поиске отличается от расшифровки, где там не допускается большая доль неправильного распознавания.
В первой главе были рассмотрены основные подходы к распознаванию иероглифических текстов, как в постановке задачи расшифровки, так и в рамках поисково-ориентированных методов. Однако эффективность и применимость этих подходов в значительной степени зависят от свойств анализируемых данных. В случае древнекитайских рукописных источников сами характеристики текстового материала накладывают ряд принципиальных ограничений, которые необходимо учитывать при формулировке задачи и выборе методов её решения.

Одной из ключевых особенностей древнекитайских рукописей является невозможность формирования полного и замкнутого множества иероглифов, встречающихся в реальных архивных источниках. Современные методы распознавания китайских текстов, как правило, ориентированы на ограниченный набор часто используемых иероглифов, оцениваемый примерно в 10 000 знаков. Для этого подмножества существуют хорошо аннотированные датасеты, используемые для обучения нейросетевых моделей \cite{li2022large, sun2019chinese}. Однако при работе с историческими архивами такие данные оказываются недостаточными. С учётом устаревших и редко используемых число символов может достигать более 100 000. Подобные редкие тексты могут быть представлены лишь в одном или нескольких документах, либо полностью отсутствовать в доступных наборах данных. В результате обученные модели не способны корректно распознавать эти символы.

Процесс ручной разметки исторических рукописей является крайне трудоёмким и дорогостоящим, требующим привлечения специалистов высокой квалификации. В литературе предпринимались попытки формирования специализированных датасетов для древних иероглифов \cite{xu2019casia, ma2020joint, shi2023m5hisdoc}, но даже при значительных усилиях по созданию датасетов невозможно гарантировать покрытие всего пространства классов, что делает методы, основанные на обучении с фиксированным словарём иероглифов, концептуально ограниченными. Более того, результаты распознавания с использованием передовых нейросетевых моделей для редко встречающихся символов остаются неудовлетворительными. Так, в работе \cite{shi2023m5hisdoc} показано, что несмотря на достижение top-1 accuracy на уровне 94\%, макроточность распознавания для всех рассмотренных моделей снижается до 71\%. Такое расхождение указывает на наличие выраженной проблемы «длинного хвоста» в распределении иероглифов обучающей выборки: редкие символы, представленные ограниченным числом примеров, систематически распознаются с низкой точностью, что существенно ограничивает применимость классических методов расшифровки в условиях открытого множества классов. На рисунке \cref{fig:sec2_freq_curve} показано частотное распределение иероглифов в типичных китайских текстах. Анализ данного распределения показывает, что менее 20\% иероглифов обеспечивают около 80\% общего объёма письменной продукции, что свидетельствует о выраженном эффекте «длинного хвоста» в распределении частот иероглифов.

\begin{figure}[ht]
    \centerfloat{
        \includegraphics[width=0.8\linewidth]{sec2_freq_curve.png}
    }
    \caption[]{Частотное распределение иероглифов в типичных китайских текстах}\label{fig:sec2_freq_curve}
\end{figure}

% -------------------------
Дополнительную сложность представляет высокая вариативность начертаний, обусловленная отсутствием строгих стандартов письма, индивидуальным почерком переписчиков, физическим состоянием носителя и историческим контекстом создания документа. Примеры таких древних китайских документов показаны на рисунке \cref{fig:sec2_ancient_docs}. В результате визуальное сходство между экземплярами одного и того же иероглифа может быть ниже, чем сходство между различными знаками, что существенно снижает надёжность прямого посимвольного распознавания.

\begin{figure}[ht]
    \begin{subfigure}[t]{0.45\textwidth}
        \centering
        \includegraphics[width=\linewidth]{sec2_ancient_docs_1.jpg}
    \end{subfigure}
    \hfill
    \begin{subfigure}[t]{0.45\textwidth}
        \centering
        \includegraphics[width=\linewidth]{sec2_ancient_docs_2.jpg}
    \end{subfigure}
    \par\bigskip
    \hfill
    \centerfloat{
        \includegraphics[width=0.95\linewidth]{sec2_ancient_docs_3.jpg}
    }
    \caption{Примеры древнекитайских рукописей. Изображения из набора данных M5HisDoc \cite{shi2023m5hisdoc}.}
    \label{fig:sec2_ancient_docs}
\end{figure}

% -------------------
Важно отметить, что обработка древнекитайских рукописных архивов, как правило, не ориентирована на массовое применение. Задачи автоматизации обработки древних архивов представляют интерес преимущественно для узкого круга исследователей — историков, филологов и лингвистов, и потому редко становятся объектом коммерческих разработок. В этом контексте автоматические методы выступают не как замена эксперта, а как вспомогательный инструмент, облегчающий навигацию по большим коллекциям изображений. Соответственно, задачи создания обширных размеченных корпусов и обучения универсальных моделей распознавания оказываются вторичными по сравнению с задачей эффективного поиска и сопоставления.

Существенным является и различие в критериях качества. В классической задаче расшифровки текста допустимая доля ошибок крайне мала, поскольку даже единичные неверно распознанные символы могут искажать смысл документа. В задачах же поиска и навигации по архивам допускается наличие неточностей на уровне отдельных иероглифов, при условии, что система способна корректно выявлять визуально или структурно близкие фрагменты и предлагать исследователю релевантные кандидаты для последующего анализа.

Перечисленные особенности указывают на то, что многие методы, рассмотренные в главе \ref{ch:ch1}, сталкиваются с фундаментальными ограничениями при применении к древнекитайским рукописным текстам. В частности, подходы, ориентированные на полную расшифровку и обучение на больших размеченных выборках, плохо согласуются с условиями редкости данных, открытого множества классов и поисковой природы практических сценариев использования.

В связи с этим более адекватной представляется постановка задачи в терминах навигации и контекстного поиска по иероглифическим изображениям, где распознавание трактуется как задача сопоставления и поиска по образцу. Такой подход естественным образом приводит к использованию методов однократного обучения, а также поисково-ориентированных мер сходства, что и будет рассмотрено в последующих разделах данной главы.

\section{Контекстный поиск по иероглифическому запросу}\label{sec:ch2/sec2}
В рамках навигации по цифровым архивам древнекитайских рукописных текстов задача обработки иероглифических изображений естественным образом формулируется как задача контекстного поиска по визуальному запросу. В данной постановке основной целью является выявление и ранжирование фрагментов архива, релевантных заданному иероглифическому образцу, с учётом как его визуальных характеристик, так и окружения, в котором он встречается.

Пусть цифровой архив представлен набором изображений рукописных документов

\[
    D = \{I_1, I_2, \ldots, I_N \}
\]

где каждый документ состоит из последовательности иероглифических фрагментов или отдельных иероглифов, полученных в результате предварительной сегментации и предобработки. Запросом в задаче контекстного поиска является изображение отдельного иероглифа

\[
    q \in Q
\]

которое может быть взято непосредственно из исследуемого корпуса либо задано пользователем в виде эталонного образца. Важно отметить, что для запроса, как правило, доступно лишь одно или несколько наблюдений, что соответствует однократному сценарию.

В отличие от классической задачи распознавания, где требуется однозначно сопоставить запросу символьную метку из фиксированного словаря, в задаче контекстного поиска предполагается открытое множество классов. Запрос может соответствовать редкому, ранее не наблюдавшемуся или неоднозначно интерпретируемому иероглифу, что делает невозможным использование строгой классификации.

Задача контекстного поиска формулируется следующим образом: для заданного запроса $q$ требуется определить подмножество фрагментов.

\[
    R(q) \in D
\]

которые являются релевантными данному запросу, и упорядочить их по степени сходства. При этом релевантность не требует точного графического совпадения или однозначного семантического соответствия, а определяется степенью визуального и структурного сходства, достаточной для последующего экспертного анализа.

Иллюстрация постановки задачи контекстного поиска представлена на рисунке \cref{fig:sec2_contextual_search}. Таким образом, в отличие от задач расшифровки, где даже единичные ошибки критичны, в контекстном поиске допускается наличие неточностей на уровне отдельных совпадений. Ключевым требованием является способность системы эффективно сузить пространство поиска и предоставить пользователю ограниченный набор наиболее вероятных кандидатов.

\begin{figure}[ht]
    \centerfloat{
        \includegraphics[width=1.0\linewidth]{sec2_contextual_search.jpeg}
    }
    \caption[]{Постановка задачи контекстного поиска}\label{fig:sec2_contextual_search}
\end{figure}

С учётом особенностей древнекитайских рукописных текстов, описанных в разделе \ref{sec:ch2/sec1}, контекстный поиск по иероглифическому запросу представляет собой более адекватную и практически значимую постановку задачи. Такая формализация естественным образом приводит к использованию поисково-ориентированных методов сопоставления изображений, а также подходов однократного обучения, что будет подробно рассмотрено в следующем разделе.

\section{Предлагаемые методы решения задачи}\label{sec:ch2/sec3}
% 1. Предлагаем рассмотрение конечного набора точек в качестве признакового описания при сравнении.
% 2. Предлагаем использовать решение задачи о назначениях в качестве меры расходства между иероглифами.

Исходя из особенностей древнекитайских рукописных текстов, рассмотренных в разделе \ref{sec:ch2/sec1}, и формализации задачи контекстного поиска, описанной в разделе \ref{sec:ch2/sec2}, в данной работе рассматривается поисково-ориентированный подход к анализу иероглифических изображений. Основной целью является построение системы, способной эффективно сопоставлять визуальные образы иероглифов в условиях ограниченного числа примеров, высокой вариативности начертаний и открытого множества классов.

Предлагаемый подход базируется на представлении иероглифического изображения в виде компактного и информативного признакового описания с использованием морфологическим анализом и последующем прямом вычислении меры сходства между запросом и элементами архива через метод линейного программирования. В соответствии с данной логикой методы, разработанные в рамках диссертации, условно разделены на две взаимосвязанные части, каждая из которых будет подробно рассмотрена в последующих главах.

Первая часть посвящена построению признакового описания иероглифических изображений. Основное внимание уделяется извлечению устойчивых геометрических и структурных характеристик, отражающих форму и топологию начертания. Предложенные методы основаны на морфологическом анализа изображения иероглифа и построение признакового описания через набор конечного критичных точек. Подробное описание используемых методов и этапов формирования признакового пространства приводится в главе \ref{ch:ch3}.

Вторая часть посвящена построению меры сходства между иероглифическими изображениями, сформулированной в контексте задачи поиска. Здесь задача сопоставления трактуется как оптимизационная проблема в виде линейного программирования, позволяющая сравнить сходства между точечными признаками, предлагаемыми в данной диссертации. Использование модели линейного программирования обеспечивает эффективное вычисление при получении меры сходства. Разработанные модели направлены на обеспечение устойчивого ранжирования кандидатов в условиях сценариев однократного обучения и подробно рассматриваются в главе \ref{ch:ch4}.

\begin{figure}[ht]
    \centerfloat{
        \includegraphics[width=0.7\linewidth]{sec2_methods_structure.jpg}
    }
    \caption[]{Схема предложенного двухэтаного подхода для сопоставления иероглифических изображений.}\label{fig:sec2_methods_structure}
\end{figure}

Схема предложенного подхода представлена на рисунке \cref{fig:sec2_methods_structure}. Разделение предложенного подхода на указанные компоненты отражает общую концепцию работы: отделение этапа формирования признакового представления от этапа вычисления сходства позволяет гибко адаптировать систему к различным условиям поиска и облегчает анализ влияния отдельных компонентов на итоговое качество. В совокупности данные методы формируют основу для практической реализации контекстного поиска по иероглифическому запросу, примеры применения которого будут продемонстрированы в главе \ref{ch:ch5}.

\section{Выводы к главе \ref{ch:ch2}}
В данной главе была рассмотрена задача навигации по цифровым архивам древнекитайских рукописных текстов с акцентом на анализ специфики исходных данных и адекватную постановку задачи обработки исторических иероглифических изображений. Показано, что особенности древних рукописей, такие как открытое множество иероглифов, выраженное распределение с длинным хвостом, высокая вариативность начертаний и ограниченность размеченных данных, существенно ограничивают применимость классических методов полной расшифровки.

На основе данного анализа сделан вывод о целесообразности формулировки задачи в терминах поиска и навигации, где распознавание иероглифов рассматривается не как цель, а как вспомогательный инструмент для выявления релевантных фрагментов архива. Введено понятие контекстного поиска по иероглифическому запросу, позволяющее учитывать как визуальные особенности символов, так и условия их появления в документе.

В рамках выбранной постановки была представлена общая концепция предлагаемых методов, основанных на сценариях однократного и малократного обучения сопоставления изображений. Предложенный подход включает построение структурного признакового описания и разработку поисково-ориентированной меры сходства, что соответствует практическим требованиям анализа древнекитайских рукописных источников.

Полученные в данной главе выводы служат методологической основой для последующего изложения. В главах \ref{ch:ch3} и \ref{ch:ch4} подробно рассматриваются соответствующие компоненты предложенного подхода, а в главе \ref{ch:ch5} демонстрируются его практические применения в задачах однократного распознавания и поиска по рукописным документам.

\FloatBarrier
