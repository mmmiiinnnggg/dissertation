\chapter{Практические приложения контекстного поска}\label{ch:ch4}
Предлагаемый метод был реализован и экспериментально проверен. Проведенные вычислительные эксперименты направлены на оценку его работоспособности и эффективности с точки зрения использования для решения задач поиска иероглифов в больших файлах исторических архивов.

Следует отметить, что требования, предъявляемые к алгоритмам распознавания для поиска иероглифов, отличаются от требований к алгоритмам расшифровки текста, которые решают задачу перевода его в машиночитаемый формат. Алгоритмы расшифровки должны обеспечить баланс ошибок ложного узнавания (FAR – False Acceptance Rate) и пропуска цели (FRR – False Rejection Rate). Для алгоритмов поиска редких иероглифов эти критерии не равнозначны. Уровень ошибок FAR допускается существенно более высокий, чем FRR. Это объясняется тем, что найденные в результате поиска иероглифы должны пройти визуальный контроль эксперта асессора, который легко может быть реализован в диалоговом режиме. В процессе такого контроля эксперт может отвергнуть ошибочно найденные иероглифы-кандидаты. С этой работой он легко справится даже если среди найденных алгоритмом кандидатов доля правильных составит 5-10\%. В то же время пропуск цели при решении задачи поиска может оказаться фатальным, поскольку пропущенный иероглиф в большом файле будет потерян.

Результат работы алгоритма поиска состоит в формировании списка иероглифов кандидатов, который передаётся эксперту для визуального анализа. Это ранжированный список символов файла по убыванию их сходства с запросом.  Поэтому успех поиска определяется вероятностью попадания в топ этого списка тех иероглифов из файла, которые совпадают с иероглифом из запроса. В соответствии с этими соображениями для оценки алгоритма поиска в качестве критерия используются оценки этой вероятности в зависимости от размера топовой части списка.

\section{Методы ускорения алгоритма}\label{sec:ch4/sect0}
\subsection{Распараллеливание сравнений изображений}\label{subsec:ch4/sect0/sub1}

TODO:

\subsection{Предварительный подсчет признаков изображений}\label{subsec:ch4/sect0/sub2}

TODO:

\section{Задача однократного распознавания}\label{sec:ch4/sect1}
Чтобы сравнение предлагаемого метода с другими подходами было справедливым и масштабированным, мы проводили эксперименты однократного распознавания на большом аннотированном китайском рукописном корпусе CASIA-AHCDB \cite{xu2019casia} с 2.2 миллионами примеров иероглифов из более 10000 классов. Задача однократного распознавания состоит в нахождении самого похожего иероглифа среди набора примеров-кандидатов с определенным количеством (5-20 штук) при заданном иероглифе запроса. В набор кандидатов входит один положительный пример, который является другим изображением иероглифа запроса, и несколько отрицательных примеров, на которых изображены другие иероглифы. Если самый похожий иероглиф, выданный алгоритмом, является положительным примером, то считается, что для данной задачи эксперимент дал положительный выход. Согласно \cite{liu2022one}, мы проводили по 550 экспериментов с рандомно взятыми изображениями иероглифов из большого размеченного набора данных и посчитали итоговую точность классификации по следующей формуле:
\[
    \text{Accuracy} = \frac{\text{Number of experiments with positive outcome}}{\text{Total number of experiments}}
\]

Все вычислительные эксперименты проводились на обычном ноутбуке. Результат экспериментов представлен в таблице 1. Мы выбрали работы \cite{liu2022one,li2019one} в качестве базовых методов для сравнения, поскольку это последние передовые методы, использующие нейронные сети, и мы проводили эксперименты в той же самой конфигурации, что и в этих работах. Для каждого эксперимента мы использовали 5 и 20 изображений для формирования набора кандидатов в соответствии с \cite{liu2022one}, которые в таблице \cref{tab:result_1} обозначены «5-way» и «20-way». Стоит отметить, что несмотря на то, что предложенный метод реализует алгоритмический подход, его точность классификации сопоставима или даже превышает точность последних передовых методов, использующих нейронные сети и мощные вычислительные устройства. Это свидетельствует об эффективности нашего метода.

\begin{table}
    \centering
    \captionsetup{justification=centering} % выравнивание подписи по-центру
    \caption{Сравнение точности классификации методов однократного распознавания}\label{tab:result_1}
    \begin{tabular}{cccc}
        \toprule
        \textbf{Метод}                   & \textbf{Категория}       & \textbf{5-way} & \textbf{20-way} \\
        \midrule
        Сиамская сеть \cite{li2022large} & Нейросетевой             & 95.00          & -               \\
        MED \cite{ma2020joint}           & Нейросетевой             & 96.64          & 94.52           \\
        \textbf{Предлагаемый метод}      & \textbf{Алгоритмический} & \textbf{96.91} & \textbf{95.10}  \\
        \bottomrule
    \end{tabular}
\end{table}

\section{Поисковый запрос в рукописных документах}\label{sec:ch4/sect2}

Для того, чтобы оценить эффективность предлагаемого метода в реальном сценарии поиска иероглифов в рукописных документах, мы взяли отсканированные страницы из набора документов MTHv2 \cite{ma2020joint}. Примеры страниц показаны на рисунках \cref{fig:real_manuscript_1,fig:real_manuscript_2}. Видно, что в документах присутствуют помехи и шумы разного уровня из-за долгого времени хранения в архивах и нестабильного качества сканирования. Задача поиска состоит в нахождении всех иероглифов в представленных документах при заданном запросе. Качество поиска определяется процентом вхождения истинных иероглифов в топ списка, ранжированного по величине меры сходства запроса и иероглифов из файла. Чем больше процент, тем меньше ошибки пропуска цели, то есть пропуска иероглифов, совпадающих с запросом при поиске. Для экспериментов мы взяли 4 документа, в которых подобрали те иероглифы, у которых число экземпляров изображений превышает 20. В сумме получилось 1600 иероглифов.
\begin{figure}[ht]
    \centerfloat{
        \includegraphics[width=0.6\linewidth]{real_manuscript_1.jpg}
    }
    \caption[Пример страниц набора документов MTHv2, первый документ]{Пример страниц набора документов MTHv2, первый документ.}\label{fig:real_manuscript_1}
\end{figure}
\begin{figure}[ht]
    \centerfloat{
        \includegraphics[width=0.6\linewidth]{real_manuscript_2.jpg}
    }
    \caption[Пример страниц набора документов MTHv2, второй документ]{Пример страниц набора документов MTHv2, второй документ.}\label{fig:real_manuscript_2}
\end{figure}

Для каждого иероглифа мы осуществили поиск среди всех 1600 изображений с одним произвольно взятым экземпляром в качестве запроса, и итоговый результат получен путём усреднения процентов вхождения для всех долей (от Top5\% до Top50\%) по всем иероглифам. Результат экспериментов показан на таблице \cref{tab:result_2}. Из таблицы видно, что при отборе в топ первых 10\% иероглифов в списке предлагаемый метод может локализовать из них истинные иероглифы с точностью больше 90\%. Это означает, что конечные пользователи нашего предложенного метода могут отфильтровать документы и сократить усилия рассмотрения и изучения материалов по меньшей мере в 10 раз, при этом существенные потери в поиске иероглифов не обнаруживаются.
\begin{table}
    \centering
    \captionsetup{justification=centering} % выравнивание подписи по-центру
    \caption{Результат поиска иероглифов при разных долях упорядоченного списка вычисленных мер сходства между запросом и изображениями из файла}\label{tab:result_2}
    \begin{tabular}{ccccc}
        \toprule
        \textbf{Доля}     & \textbf{Top5\%} & \textbf{Top10\%} & \textbf{Top20\%} & \textbf{Top50\%} \\
        \midrule
        Процент вхождения & 81.25           & 90.50            & 95.50            & 98.75            \\
        \bottomrule
    \end{tabular}
\end{table}


\section{Выводы к главе \cref{ch:ch4}}
\clearpage