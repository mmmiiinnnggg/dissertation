\chapter{Методы построения меры сходства изображений иероглифических текстов}\label{ch:ch4}
% Из ISPRS
Настоящая глава посвящена следующему ключевому этапу предлагаемого подхода — построению методов сравнения иероглифов на основе их точечных представлений. Поскольку каждому изображению иероглифа сопоставляется конечное множество точек, задача сравнения двух изображений естественным образом формулируется как задача сопоставления двух точечных конфигураций в евклидовом пространстве. В этом контексте распознавание иероглифов трактуется не как задача классификации с фиксированным набором классов, а как поисковая задача, ориентированная на вычисление меры сходства и последующее ранжирование элементов файла относительно заданного запроса.

В отличие от традиционных методов, предполагающих наличие большого количества обучающих примеров для каждого класса, рассматриваемая постановка ориентирована на работу в условиях открытого множества классов и ограниченного или нулевого числа образцов. Это делает невозможным прямое применение стандартных классификационных моделей и требует разработки специализированных мер сходства, способных учитывать как геометрические, так и структурные различия между точечными представлениями иероглифов.

Сложность сравнения состоит в том, что созвездия изображений одного и того же иероглифа часто оказываются неизоморфными графами. Это происходит из-за естественных различий в начертании рукописных символов, деформации древних рукописей вследствие длительного хранения, а также искажений при сканировании документов. Деформации и искажения выражаются в том, что на полученных цифровых изображениях некоторые штрихи слипаются, либо в них появляются разрывы. Поэтому топологический критерий сходства метаграфов недостаточен сравнения иероглифов.

Предлагаемая мера близости созвездий определяется на основе построения наилучшего паросочетания точек между точками созвездий. Критерий качества паросочетания может быть выбран различными способами. Мы рассматриваем несколько вариантов такого выбора. Все варианты критерия приводят к оптимизационной задаче о назначениях.

\section{Задача о назначениях и линейное программирование}\label{sec:ch4/sec0}
Задача о назначениях является одной из базовых задач дискретной оптимизации и линейного программирования и служит стандартной моделью для формализации процессов сопоставления элементов двух конечных множеств. В широком смысле она описывает ситуации, в которых требуется установить соответствие между объектами при минимизации суммарной стоимости такого соответствия, либо максимизации показателя качества соответствия, в частности числа допустимых парных сочетаний. Благодаря строгой математической постановке, полиномиальной разрешимости и наличию эффективных алгоритмов, задача о назначениях широко применяется как в теоретических исследованиях, так и в прикладных задачах анализа данных, компьютерного зрения и распознавания образов.

Пусть заданы два конечных множества объектов
\[
    A = \{a_1, a_2, \cdots, a_n\}, B = \{b_1, b_2, \cdots, b_m \}
\]
между элементами которых требуется установить взаимно-однозначное соответствие. Предполагается, что каждому возможному назначению $a_i \leftrightarrow b_i$ сопоставлена числовая стоимость $c_{ij}\in\mathbb{R}$, характеризующая степень несоответствия или «цену» данного назначения. Совокупность таких стоимостей образует матрицу
\[
    C = (c_{ij})^{n\times m}_{i,j=1}
\]
Введём бинарные переменные
\[
    x_{ij} =
    \begin{cases}
        1, & \text{если объект } a_i \text{ назначен объекту } b_j, \\
        0, & \text{в противном случае}.
    \end{cases}
\]
Тогда классическая задача о назначениях формулируется следующим образом:
\[
    \begin{aligned}
         & \min_{x_{ij}}       &  & \sum_{i=1}^{n} \sum_{j=1}^{m} c_{ij} x_{ij}, \\
         & \text{при условиях} &  &
        \sum_{j=1}^{m} x_{ij} = 1, \quad i = 1, \dots, n,                        \\
         &                     &  &
        \sum_{i=1}^{n} x_{ij} = 1, \quad j = 1, \dots, m,                        \\
         &                     &  &
        x_{ij} \in \{0,1\}.
    \end{aligned}
\]
Эти ограничения обеспечивают биективность соответствия: каждому элементу множества $A$ ставится в соответствие ровно один элемент множества $B$, и наоборот.

Несмотря на наличие бинарных переменных, задача о назначениях обладает важным структурным свойством: её линейная релаксация, получаемая заменой условий $x_{ij}\in\{0, 1\}$ на
\[
    x_{ij} \geq 0.
\]
имеет целочисленные оптимальные решения \cite{lovasz1975ratio}. Данное свойство позволяет рассматривать задачу о назначениях как частный случай задачи линейного программирования и применять к ней как специализированные, так и универсальные методы решения задач линейного программирования. В то же время наличие целочисленного оптимума гарантирует, что решение линейной релаксации соответствует корректному назначению.

Задача о назначениях и родственные ей модели находят применение в широком спектре прикладных областей, включая:
\begin{itemize}
    \item распределение ресурсов и планирование;
    \item логистику и транспортные задачи;
    \item сопоставление вершин графов и подграфов;
    \item регистрацию изображений и сопоставление признаков;
    \item анализ формы объектов и структурных представлений данных.
\end{itemize}
Особое значение эта модель имеет в задачах компьютерного зрения и распознавания образов, где требуется установить соответствие между ключевыми точками, примитивами или локальными признаками двух объектов. В таких задачах матрица стоимостей $C$ обычно формируется на основе геометрических расстояний, угловых различий или иных локальных характеристик.

Наиболее известным специализированным алгоритмом для решения задачи о назначениях является венгерский алгоритм \cite{kuhn1955hungarian}, обеспечивающий точное решение за $O(n^3)$ время. Альтернативные подходы включают методы на основе потоков минимальной стоимости, а также универсальные алгоритмы линейного программирования, такие как симплекс-метод и внутренние точечные методы. Последние обладают полиномиальной теоретической сложностью и демонстрируют высокую практическую эффективность для задач умеренного размера.

Важно отметить, что задача о назначениях относится к числу хорошо изученных задач оптимизации, для которых существуют зрелые и высокоэффективные программные реализации. Современные LP-солверы и специализированные библиотеки \cite{mitchell2011pulp} позволяют надёжно и быстро решать такие задачи даже при достаточно больших размерах входных данных, что делает их привлекательными для использования в прикладных системах поиска и сравнения.

В контексте настоящей работы каждый иероглиф представляется в виде конечного множества характерных точек — созвездия, описывающего его геометрическую структуру. Сравнение двух иероглифов, таким образом, сводится к задаче сопоставления двух конечных точечных множеств. Данная идея иллюстрируется на рисунке \cref{fig:sec4_overlapping_constellations}, где при наложении созвездий двух иероглифов формируется паросочетание точек. Это сопоставление естественным образом формулируется как задача о назначениях, где элементы одного созвездия должны быть сопоставлены элементам другого с минимальными суммарными затратами, определяемыми геометрическим и структурным несоответствием.

\begin{figure}[ht]
    \centerfloat{
        \includegraphics[width=0.9\linewidth]{sec4_overlapping_constellations.jpg}
    }
    \caption{Созвездия иероглифов и паросочетание минуций при наложении.}\label{fig:sec4_overlapping_constellations}
\end{figure}

Данная модель обладает рядом существенных преимуществ: она не требует строгого совпадения структуры, допускает частичное соответствие и обеспечивает формальную, оптимизационно обоснованную меру сходства. Наличие эффективных и проверенных на практике алгоритмов решения делает задачу о назначениях естественным и методологически обоснованным выбором для сравнения созвездий иероглифов в поисково-ориентированной постановке, принятой в данной диссертационной работе.

\section{Модель минимальной стоимости}\label{sec:ch4/sec1}
В настоящем разделе рассматривается базовая модель сопоставления созвездий, основанная на минимизации суммарной стоимости соответствий между их точками. Данная модель является наиболее прямым и естественным применением задачи о назначениях, введённой в предыдущем разделе, к задаче сравнения точечных представлений иероглифов. Она служит отправной точкой для построения более сложных и устойчивых моделей сопоставления, рассматриваемых далее.

Основная идея модели заключается в следующем: два иероглифа считаются тем более сходными, чем меньшей является суммарная геометрическая разница между точками их созвездий при оптимальном выборе соответствий. Таким образом, задача сравнения двух иероглифов формулируется как задача поиска такого паросочетания между точками двух созвездий, при котором суммарная стоимость соответствий минимальна.

Формально, пусть $n,m$ – число точек в двух созвездиях. Обозначим $X=\{x_{ij}\}$ – бинарные переменные, указывающие соответствие между $i$-ой точкой первого созвездия и $j$-ой точкой второго созвездия. Если $x_{ij}=1$, то соответствие между этими точками установлено, если $x_{ij}=0$, то нет. Пусть $\lambda_{ij}$ – коэффициенты стоимости, определяющие вклад соответствия пары точек $(i,j)$ в общую меру различия между созвездиями. В простейшем случае такие коэффициенты могут быть заданы евклидовым расстоянием между соответствующими точками, однако в общем случае они могут учитывать и другие геометрические или структурные характеристики.

Оптимизационная задача о назначениях имеет следующий вид:
\[
    \min f(X) = \sum_{i=1}^{n}\sum_{j=1}^{m}\lambda_{ij}x_{ij}, \quad X =\{x_{ij}\} \in \{0,1\}, \sum_{i=1}^{n}x_{ij}\ge 1, \sum_{j=1}^{m}x_{ij}\ge 1.
\]
Целевая функция $f(X)$ представляет собой суммарную стоимость всех выбранных соответствий и служит количественной мерой различия двух созвездий. Чем меньше значение этой функции в оптимуме, тем меньшей считается геометрическая и структурная разница между сравниваемыми иероглифами.

Наложенные ограничения на переменные $x_{ij}$ обеспечивают участие каждой точки обоих созвездий в процессе сопоставления. Условие
\[
    \sum_{i=1}^{n}x_{ij}\ge 1
\]
гарантирует, что каждая точка первого созвездия имеет по крайней мере одну соответствующую точку во втором, а аналогичное условие
\[
    \sum_{j=1}^{m}x_{ij}\ge 1
\]
обеспечивает участие каждой точки второго созвездия. Таким образом, ни одна точка не исключается из процедуры сравнения, что позволяет учитывать вклад всей структуры иероглифа при вычислении меры сходства.

Использование неравенств вместо строгих равенств принципиально отличает данную модель от классической задачи о назначениях. Такое ослабление ограничений допускает установление множественных соответствий и позволяет моделировать ситуации, в которых локальные деформации, шумы или различия в детализации скелетного представления приводят к неоднозначным соответствиям между точками. В контексте сравнения иероглифов это особенно важно, поскольку скелетные структуры разных экземпляров одного и того же символа, как правило, не являются строго изоморфными.

С геометрической точки зрения минимизация функции $f(X)$ соответствует поиску наилучшего согласования двух точечных конфигураций, при котором суммарное расстояние между соответствующими элементами минимально. В отличие от локальных мер сходства, основанных на поиске ближайших соседей, данная модель учитывает глобальную согласованность соответствий и тем самым обеспечивает более устойчивую оценку сходства формы.

Рассматриваемая оптимизационная задача относится к классу целочисленных линейных программ и в общем случае имеет $O(nm)$ бинарных переменных. Благодаря линейной структуре целевой функции и ограничений, она допускает эффективное решение с использованием стандартных методов линейного программирования после релаксации целочисленных ограничений, а также специализированных алгоритмов, разработанных для задач о назначениях. Практическая применимость данной модели дополнительно обеспечивается наличием зрелых и высокоэффективных программных решателей, позволяющих получать оптимальные решения за приемлемое время при размерах созвездий, характерных для рассматриваемой задачи.

В то же время следует отметить, что модель минимальной стоимости обладает рядом ограничений. В частности, она предполагает обязательное участие каждой точки в сопоставлении, что может приводить к избыточным соответствиям в присутствии шумовых или артефактных точек. Кроме того, возможность множественных соответствий может снижать дискриминативность меры сходства в случаях существенного структурного различия созвездий.

Тем не менее, данная модель представляет собой естественную и методологически обоснованную базовую постановку задачи сравнения созвездий. Она обладает простотой, интерпретируемостью и хорошей вычислительной реализуемостью, что делает её удобной отправной точкой для дальнейших модификаций.

\section{Модель максимального количества близких пар}\label{sec:ch4/sec2}
Другая модель максимизирует меру сходства созвездий. Она состоит в выборе максимального числа допустимых паросочетаний при условии, что допустимыми парами является только те, у которых евклидово расстояние $\lambda_{ij}$ между точками не превосходит заданного порога, и при этом каждая точка может входить не более чем в одну пару. Пусть коэффициент $\mu_{ij}$ – индикатор того, что расстояние $\lambda_{ij}$ между $i$-ой точкой первого созвездия и $j$-ой точкой второго созвездия не превышает порога $p$, т.е. $\mu_{ij}=\mathbb{I}[\lambda_{ij}\le p]$.

Здесь тоже получается задача о назначениях, которая может быть записана в следующем виде:
\[
    \max f(X) = \sum_{i=1}^{n}\sum_{j=1}^{m}\mu_{ij}x_{ij}, \quad X =\{x_{ij}\} \in \{0,1\}, \sum_{i=1}^{n}x_{ij}\le 1, \sum_{j=1}^{m}x_{ij}\le 1.
\]
Величина $\frac{f(X)}{mn}$ , полученная на основе решения этой задачи, используется как оценка сходства между созвездиями двух иероглифов. Использование нормирующей величины $mn$ обеспечивает независимость принимаемого решения от числа вершин, входящих в созвездия. Для решения задачи о назначениях в обеих моделях мы используем алгоритм, реализованный в библиотеке \verb|PuLP| \cite{mitchell2011pulp}.

При решении задачи поиска иероглифов в файле по заданному запросу вычисляется мера сходства и различия запроса со всеми иероглифами файла. В случае использования первой модели составляется ранжированный список иероглифов из файла по возрастанию различий между ними и запросом. При использовании второй модели получается ранжированный список по убыванию сходства запроса и иероглифов из файла. Первые по порядку иероглифы в этих списках будут рассматриваться как кандидаты для решения задачи поиска. Они предъявляются исследователю в качестве вариантов решения задачи.

\section{Топологическая согласованность при сопоставлении созвездий}\label{subsec:ch4/sec3}
В предыдущих подразделах была рассмотрена процедура построения созвездий и нахождения их оптимального соответствия на основе метрической близости между элементами. Полученная таким образом мера сходства учитывает в основном геометрические и локальные структурные характеристики отдельных узлов. Однако для графовых представлений, полученных из скелетных структур древних иероглифов, не менее важным является сохранение \emph{топологической согласованности} между сопоставляемыми элементами.

% \paragraph{Мотивация.} 
Интуитивно, если две пары узлов из разных созвездий были сопоставлены между собой, и при этом соответствующие узлы в обоих метаграфах соединены ребром, то такое сопоставление является более правдоподобным с точки зрения глобальной структуры. Напротив, если оптимальное по расстоянию сопоставление разрушает исходные связи, это может указывать на случайное или локально оптимальное, но структурно некорректное соответствие. Поэтому после выполнения этапа оптимального сопоставления возникает необходимость дополнительной проверки согласованности связей между сопоставленными узлами.

% \paragraph{Формальная постановка.} 
Пусть заданы два метаграфа $G_1 = (V_1, E_1), G_2 = (V_2, E_2)$, где вершины соответствуют элементам созвездий, а рёбра кодируют их топологические связи. В результате решения задачи оптимального назначения получаем бинарную матрицу соответствий $C=\{0,1\}^{|V_1|\times|V_2|}$, где
\[
    C_{ij} = 1 \iff v_i \in V_1 \text{ сопоставлен с } u_j \in V_2.
\]

Рассмотрим множество всех сопоставленных пар вершин:
\[
    \mathcal{M} = \{(v_i, u_j) \mid C_{ij} = 1\}.
\]
Для любых двух различных элементов $(v_i, u_j), (v_k, u_l)\in\mathcal{M}$ проверяется сохранение топологической связи:
\[
    (v_i, v_k) \in E_1 \ \wedge \ (u_j, u_l) \in E_2.
\]
Если данное условие выполнено, то считается, что соответствие между этими двумя парами узлов является \emph{топологически согласованным}.

% \paragraph{Количественная оценка связности.} 
Общее число топологически согласованных связей определяется как
\[
    N_{\text{conn}} = \sum_{(v_i,u_j),(v_k,u_l) \in \mathcal{M}} \mathbb{I}\big[(v_i,v_k)\in E_1 \land (u_j,u_l)\in E_2\big],
\]
где $\mathbb{I}[\cdot]$ - индикаторная функция. Для нормализации данной величины используется масштабный коэффициент $N_{\max} = \max(|V_1|, |V_2|)$, что позволяет интерпретировать отношение $N_{conn}/N_{max}$ как относительную степень сохранения топологии.

% \paragraph{Интеграция в итоговую меру сходства.} 
Пусть $S_0$ — исходная мера сходства, полученная на этапе оптимального сопоставления созвездий. Тогда итоговая мера с учетом топологической согласованности определяется как
\[
    S = S_0 \cdot \left(1 + \alpha_{\text{conn}} \left(1 - \frac{N_{\text{conn}}}{N_{\max}}\right) \right)
\]
где $\alpha_{comm}\ge 0$ - коэффициент, контролирующий вклад топологического штрафа. Таким образом, чем больше количество сохранённых связей между сопоставленными узлами, тем меньший штраф добавляется к итоговой мере, и тем выше считается структурное сходство двух объектов.

% \paragraph{Практическая реализация.} 
На практике описанная процедура реализуется после получения матрицы соответствий. Перебираются все комбинации сопоставленных пар вершин, и для каждой комбинации выполняется проверка наличия рёбер в обоих метаграфах. Подсчитанное значение $N_{conn}$ используется для корректировки итоговой меры сходства. Такой подход не требует повторного решения задачи сопоставления и вводит лишь линейную по числу сопоставленных узлов дополнительную вычислительную сложность.

% \paragraph{Связь с последующими разделами.} 
Введение топологического штрафа позволяет перейти от чисто локального сравнения элементов созвездий к более глобальному учету структуры. В следующем разделе будет показано, как данная модифицированная мера сходства влияет на результаты поиска и сопоставления древних иероглифических форм на реальных корпусах данных.

% \section{Сравнение и обсуждение}

\section{Выводы к главе \cref{ch:ch4}}

В данной главе диссертационной работы рассмотрены методы построения меры сходства между изображениями иероглифических текстов на основе точечных признаковых описаний, полученных с использованием непрерывных морфологических моделей. Задача сравнения иероглифов формализована как задача о назначениях, что позволило использовать аппарат линейного программирования для получения количественной и интерпретируемой оценки сходства.

В главе исследованы две оптимизационные модели сопоставления: модель минимальной стоимости и модель максимального количества близких пар. Показано, что модели обеспечивают учет геометрических различий между признаками. Рассмотрены условия их применения в задачах поиска и сопоставления иероглифических изображений.

Дополнительно введены ограничения, обеспечивающие топологическую согласованность сопоставления критических точек, что позволяет сохранять внутреннюю структуру иероглифа и повышает устойчивость меры сходства при работе с рукописными и деградированными изображениями.

В результате в главе сформирован формальный и вычислительно эффективный аппарат оценки сходства иероглифических изображений, логически дополняющий методы построения признакового описания, изложенные в главе \cref{ch:ch3}, и служащий основой для практических приложений контекстного поиска, рассматриваемых далее в диссертационной работе.

\clearpage
