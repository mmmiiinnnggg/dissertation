\chapter{Навигация в цифровых древнекитайских рукописных архивах}\label{ch:ch1plus}
\section{Особенности древнекитайских рукописных текстов}\label{sec:ch1plus/sec0}
TODO:
1. невозможность сбора полностью всех иероглифов, включающих редкие встречающие.
2. некоммерческая задача, за которую интересует только узкий кург профиссианалов. культурное наследие - миссия университетов.
3. Критерий ошибки в задаче поиске отличается от расшифровки, где там не допускается большая доль неправильного распознавания.

\section{Контекстный поиск по иероглифическому запросу}\label{sec:ch1plus/sec1}
% Из PRIA

TODO:
1. Из вышксказанного вытекает, что для поставленной задачи нужно однократное обучение (one-shot)
2. Какие недостатки у существующих методов и почему нужен новый метод




Данная проблема особенно ярко проявляется при работе с языками, обладающими большим алфавитом. Китайский язык является ярким примером такого случая: число иероглифов превышает 60 000, а с учётом устаревших и редко используемых символов может достигать более 100 000. Сбор репрезентативных обучающих наборов данных для всех возможных иероглифов в таких условиях становится практически невозможным.

Современные методы распознавания печатных китайских текстов, как правило, ориентированы на ограниченный набор часто используемых иероглифов, оцениваемый примерно в 10 000 знаков. Для этого подмножества существуют хорошо аннотированные датасеты, используемые для обучения нейросетевых моделей \cite{li2022large, sun2019chinese}. Однако при работе с историческими архивами такие данные оказываются недостаточными, поскольку значительная часть встречающихся в древних текстах иероглифов редко встречается или отсутствует в обучающих выборках. В результате обученные модели не способны корректно распознавать эти символы. В литературе предпринимались попытки формирования специализированных датасетов для древних иероглифов \cite{xu2019casia, ma2020joint, shi2023m5hisdoc}, однако результаты распознавания с использованием нейросетевых моделей для редко встречающихся символов остаются ограниченными \cite{shi2023m5hisdoc}, несмотря на существенные временные и трудовые затраты, связанные с созданием и разметкой подобных наборов данных.

Процесс ручной разметки исторических рукописей является крайне трудоёмким и дорогостоящим, требующим привлечения специалистов высокой квалификации. Более того, задачи автоматизации обработки древних архивов, являющихся частью национального культурного наследия, представляют интерес преимущественно для узкого круга исследователей — историков, филологов и лингвистов, и потому редко становятся объектом коммерческих разработок.

В связи с этим актуальной является задача поиска редких иероглифов в больших массивах сканированных рукописных документов в условиях отсутствия аннотированных данных для предварительного обучения. В обобщённой форме данная задача сводится к поиску наиболее похожих объектов среди большого числа кандидатов по заданному запросу. В литературе такой подход известен как однократное обучение (One-shot Learning) \cite{vinyals2016matching, snell2017prototypical, he2020memory}. Классическим примером данной парадигмы является задача распознавания лиц по одному изображению, широко применяемая в системах безопасности и идентификации личности. Методы однократного обучения также применялись для распознавания рукописных текстов \cite{koch2015siamese, gu2018meta, lake2015human}.

Ключевым элементом в задачах однократного обучения является выбор эффективного представления объектов и метрик для сравнения признаков. В этом контексте скелетные графы \cite{zhang1984fast, cychosz1994efficient, wang2018fully, livesu2012reconstructing} представляют собой перспективный инструмент для описания рукописных текстов и, в частности, китайских иероглифов, поскольку позволяют формализовать их топологическую и геометрическую структуру в виде вершин и рёбер. Для построения таких графов предложены различные методы, среди которых особый интерес представляют непрерывные модели скелетов \cite{siddiqi2008medial, LM2016continuous}, разработанные Л. М. Местецким. Эти модели обладают рядом преимуществ, включая математическую строгость, устойчивость к шумам и вычислительную эффективность.

В данной работе предлагается подход, основанный на прямом вычислении сходства иероглифов с использованием морфологического анализа изображений, методов вычислительной геометрии и линейного программирования, что позволяет эффективно решать задачу поиска редких иероглифов в условиях отсутствия размеченных обучающих данных.



\section{Предлагаемые методы решения задачи}\label{sec:ch1/sec2}
TODO:
1. Предлагаем рассмотрение конечного набора точек в качестве признакового описания при сравнении.
2. Предлагаем использовать решение задачи о назначениях в качестве меры расходства между иероглифами.

\section{Выводы к главе \ref{ch:ch1plus}}
Из особенностей рассматриваемой задачи следует, что для практического применения нужна постановка задачи поиска. Предлагаем эффективные методы для решения этой задачи.

\FloatBarrier
