\chapter{Методы построения точечного признакового описания иероглифических текстов}\label{ch:ch3}
В предыдущей главе была предложена общая схема решения поставленной в данной диссертационной работе задачи поиска, основанная на разделении процесса анализа на два ключевых этапа: построение признакового описания иероглифических изображений и последующее вычисление меры их сходства. Критически важным элементом этой схемы является формирование устойчивого, информативного и интерпретируемого представления формы иероглифа, позволяющего корректно сравнивать изображения в условиях ограниченного числа примеров и открытого множества классов.

Настоящая глава посвящена первому из указанных этапов — построению точечного признакового описания иероглифических изображений. В работе используется подход, основанный на непрерывных морфологических моделях и медиальном представлении формы. Иероглиф рассматривается как непрерывный геометрический объект, структура которого анализируется посредством извлечения скелетного представления и последующего выделения конечного множества характерных (критических) точек, отражающих его топологические и геометрические свойства.

Следует подчеркнуть, что одним из ключевых вкладов данной работы является сведение задачи сравнения скелетов, моделируемых неизоморфными графами, к сравнению конечных наборов точек. Предлагаемый подход отличается новизной, поскольку существующие методы сравнения скелетов, как правило, опираются на топологическое сходство, в частности на анализ графового изоморфизма. Ещё одним важным вкладом является введение непрерывного скелета, который обеспечивает инвариантность по отношению к параметрам исходного изображения, таким как масштаб, разрешение и другие характеристики.

В главе последовательно рассматриваются методы построения медиальной оси и скелетного графа, подходы к генерации признаков по аналогии с «созвездиями», а также процедуры отбора, нормализации и фильтрации ключевых точек. Полученное точечное представление служит основой для формального сравнения иероглифов и напрямую используется в следующей главе при построении поисково-ориентированных мер сходства на основе оптимизационных моделей.

Построение геометрической модели иероглифа основано на методах вычислительной геометрии, что обеспечивает математическую корректность модели, устойчивость вычислительных процедур и высокую вычислительную эффективность алгоритмов сравнения и поиска иероглифов.

\section{Непрерывная морфология и скелет}\label{sec:ch3/sec0}
В вычислительных системах изображения, как правило, представлены в растровой форме, задающей регулярную прямоугольную решётку пикселей, каждому из которых сопоставлено значение яркости либо вектор цветовых характеристик. Изображение в градациях серого может быть формально описано как двумерный числовой массив, тогда как многоканальные цветные изображения (например, основанные на цветовых моделях RGB или HSV) естественным образом представляются в виде трёхмерных массивов данных.

Вместе с тем, анализ наличия объектов любых человеческих текстов на изображении, их пространственного расположения и морфологических свойств возможен лишь при условии предварительного решения задачи сегментации, заключающейся в классификации каждого пикселя как принадлежащего объекту или фону. В этом контексте исходным представлением для задачи распознавания формы объектов в рамках настоящего исследования служат \textit{бинарные изображения}, в которых каждому пикселю сопоставляется значение 0 в случае его принадлежности фону и значение 1 — при отнесении к объекту.

Рассматриваемые в данной диссертационной работе методы анализа формы для иероглифических текстов основываются на понятии непрерывной скелета фигуры. Определение непрерывного скелета вместе со способом его получения кратко описано в этом разделе. Концепция непрерывного скелета бинарных изображений, используемая в данном исследовании, следует общей теории, описанной в \cite{LM2016continuous}.

% скелет
Фигурой называется замкнутая область на плоскости, ограниченная конечным числом непересекающихся замкнутых жордановых кривых. Для фигуры $A$ \textit{пустым кругом} будем называть замкнутое множество точек $$
    S_r(p)=\{q:q\in\mathbb{R}^2, d(p, q)\le r\}
$$
такое, что $S_r(p)\subset A$. $d(p, q)$ - евклидово расстояние между точкой $p$ и $q$ на плоскости. Для краткости круг $S_r(p)$ радиуса $r$ с центром в точек $p$ будем обозначать $S$. \textit{Максимальным пустым кругом} $S_{max}$ называется пустой круг, который не содержится ни в каком друогм пустом круге, т.е.
$$
    \forall S' \subset A, S' \neq S_{max}: S_{max} \not\subset S'.
$$

Используя понятие максимального пустого круга, определим скелет фигуры следующим образом: \textit{скелетом} фигуры $A$ является множество центров ее максимальных пустых кругов. На рисунке \cref{sec3_empty_circle_skeleton} изображен пример фигуры, её максимальные пустые круга и скелет.
\begin{figure}[ht]
    \centerfloat{
        \includegraphics[width=1.0\linewidth]{sec3_empty_circle_skeleton.jpg}
    }
    \caption[]{Фигура и ее граница (левое изображение), максимальные пустые круга (среднее изображение) и скелет (правое изображение).}\label{fig:sec3_empty_circle_skeleton}
\end{figure}

На скелете определяется радиальная функция $R(x,y)$, которая ставит в соответствие каждой точке скелета $(x,y)$ значение радиуса максимального пустого круга с центром в этой точке.

% аппроксимация бинарного изображения многоугольной фигурой
Для построения непрерывного скелета на первом этапе требуется сформировать многоугольную аппроксимацию исходной фигуры бинарного изображения. Выбор многоугольного представления обусловлен возможностью применения высокоэффективных алгоритмов вычислительной геометрии \cite{LM2016continuous}, обеспечивающих как построение самой аппроксимации, так и последующее формирование скелетного графа бинарного изображения.

Границу многоугольной фигуры целесообразно рассматривать как объединение конечного числа подмножеств, называемых сайтами: сайтов-точек, соответствующих вершинам многоугольника, и сайтов-сегментов, представляющих его стороны без учёта концевых точек. Скелет многоугольной фигуры (рисунок \cref{fig:sec3_multiangle_figure}) имеет структуру геометрического графа, рёбрами которого являются отрезки прямых линий и квадратичных парабол, а вершинами — их концевые точки. Каждое ребро скелета представляет собой связное множество центров вписанных окружностей, касающихся одной и той же пары сайтов, называемых образующими сайтами данного ребра. В случае, когда образующие сайты являются однотипными (две точки или два сегмента), соответствующее ребро имеет вид прямолинейного отрезка. Если же сайты разнотипны (сайт-точка и сайт-сегмент), ребро принимает форму отрезка квадратичной параболы.

\begin{figure}[ht]
    \centerfloat{
        \includegraphics[width=0.6\linewidth]{sec3_multiangle_figure.jpg}
    }
    \caption[]{Многоугольная фигура и её скелет. }\label{fig:sec3_multiangle_figure}
\end{figure}

Таким образом, непрерывный скелет многоугольника, аппроксимирующего фигуру бинарного изображения, в рамках настоящей работы рассматривается как базовое представление формы при анализе иероглифических древнекитайских текстов. Использование именно такого скелетного описания позволяет свести задачу анализа формы к исследованию свойств геометрического графа, структура которого однозначно определяется границей аппроксимирующего многоугольника.

В отличие от традиционных методов скелетизации, основанных на итеративных морфологических операциях \cite{saeed2010k3m} или дискретных преобразованиях растрового изображения \cite{delgado2014skeletonization}, рассматриваемый подход характеризуется высокой вычислительной эффективностью, поскольку оперирует компактным множеством геометрических примитивов и использует алгоритмы вычислительной геометрии с гарантированными асимптотическими оценками сложности. Существенным преимуществом непрерывного скелета является также его инвариантность по отношению к разрешению и размера изображения: результаты построения определяются исключительно геометрией аппроксимирующей фигурой и не зависят от плотности пиксельной решётки исходного изображения.

Кроме того, представление скелета в виде геометрического графа открывает возможность непосредственного применения развитого аппарата теории графов и топологических методов анализа. Это позволяет формализовать структурные характеристики иероглифов, такие как ветвление, связность и относительное расположение элементов, и тем самым создать единое математически строгое основание для последующих задач сравнения и распознавания древнекитайских иероглифических образов.

\section{Генерация признаков по аналогии созвездий}\label{sec:ch3/sec1}
% Из ММРО/PRIA

Предлагаемый в работе подход направлен на построение информативного и компактного признакового описания структуры иероглифа в виде конечного множества точек малого размера, далее называемого созвездием иероглифа. Созвездие формируется на основе медиальной оси изображения иероглифа, также называемой скелетом, который представляется в виде планарного геометрического графа. Полученное множество точек используется в качестве признакового описания при последующем сравнении иероглифических изображений в задачах поиска.

Скелетные представления широко применяются в задачах распознавания формы объектов на изображениях \cite{siddiqi2008medial}, поскольку они эффективно отражают топологическую структуру и геометрию формы. В контексте распознавания рукописных текстов скелет часто интерпретируется как след движения пишущего инструмента при формировании символа. Для изображений иероглифов скелет, как правило, представляет собой геометрический граф сложной структуры с множеством ветвлений, пересечений и локальных деформаций.

Однако особенности индивидуального почерка, а также искажения, вызванные деградацией носителя, шумами сканирования и бинаризации, приводят к тому, что скелеты изображений одного и того же иероглифа, как правило, не являются изоморфными графами. Более того, различия могут проявляться как в топологии (появление или исчезновение ветвей), так и в геометрии рёбер и вершин. Это существенно ограничивает возможность прямого графового сопоставления и делает использование строгих методов проверки изоморфизма или графового редактирования вычислительно неустойчивым и плохо масштабируемым для задач поиска.

В связи с этим в данной работе выдвигается гипотеза о том, что для поисковых задач не требуется полное сохранение всей структуры скелетного графа. Вместо этого достаточно выделить конечное подмножество характерных точек, которые в совокупности сохраняют значимую информацию о форме иероглифа и позволяют сравнивать графически сходные и различные символы. Такой переход можно рассматривать как редукцию исходного скелетного графа к компактному точечному представлению, более устойчивому к локальным искажениям и вариативности начертания.

Методы сравнения формы объектов, основанные на анализе дискретных множеств точек, находят применение при решении прикладных задач. Одним из примеров подобного подхода может служить задача идентификации созвездий на звёздном небе. Созвездие представляет собой совокупность звёзд, визуально образующих устойчивую конфигурацию на небесной сфере. Современные астрономические каталоги выделяют 88 созвездий, каждое из которых характеризуется уникальной структурой, определяемой взаимным расположением входящих в него звёзд. В состав каждого созвездия входит от 10 до 140 звёзд с видимой яркостью до 6-й звёздной величины. Процесс распознавания созвездий основывается на выделении небольшого подмножества звёзд, формирующих наиболее узнаваемый фрагмент конфигурации. Так, например, на рисунке \cref{fig:sec3_constellation_1} созвездие Большая Медведица, включающее около 126 звёзд, различимых невооружённым глазом, легко идентифицируется по семи наиболее ярким звёздам, образующим фигуру Большого Ковша — так называемый астеризм, традиционно выделяемый как самостоятельная группа на основе этой аналогии. Термин «метод созвездий» в настоящей работе используется применительно к сравнению иероглифов, поскольку обе задачи демонстрируют очевидное сходство в структурной организации и принципах распознавания.

\begin{figure}[ht]
    \centerfloat{
        \includegraphics[width=0.8\linewidth]{sec3_constellation_1.jpg}
    }
    \caption[Пример модели для сравнения формы объектов на основе конечных малых наборов их точек: созвездие Большая Медведица.]{Пример модели для сравнения формы объектов на основе конечных малых наборов их точек: созвездие Большая Медведица.}\label{fig:sec3_constellation_1}
\end{figure}

Другая хорошо известная модель сравнения формы объектов по конечным небольшим наборам их точек используется в дактилоскопии. Пример такой модели показан на рисунке \cref{fig:sec3_constellation_2}. Для идентификации личности на изображении отпечатка пальцев выделяются наборы точек, называемые минуциями. Набор минуций включает несколько десятков точек и является признаковым описанием изображения отпечатка пальца. Сравнение отпечатков выполняется с помощью специальной меры, которая вычисляется на основе наложения карт минуций и вычисления парных расстояний между ними.

\begin{figure}[ht]
    \centerfloat{
        \includegraphics[width=0.4\linewidth]{sec3_constellation_2.jpg}
    }
    \caption[Пример модели для сравнения формы объектов на основе конечных малых наборов их точек: отпечатка пальцев и минуций]{Пример модели для сравнения формы объектов на основе конечных малых наборов их точек: отпечатка пальцев и минуций.}\label{fig:sec3_constellation_2}
\end{figure}

По аналогии с указанными примерами, в данной работе предполагается, что структура иероглифа также может быть адекватно описана конечным набором характерных точек, извлечённых из его скелетного представления. Такое точечное описание не требует строгого совпадения топологии графов, обладает большей устойчивостью к локальным деформациям и естественным образом подходит для формулировки задачи поиска как задачи сопоставления двух конечных множеств точек.

Ключевым вопросом данного подхода является достаточность набора точек, входящих в созвездие, для выявления сходства созвездий, построенных по различным изображениям одного и того же иероглифа, при одновременной способности различать созвездия, соответствующие разным иероглифам. Наличие такого свойства может быть установлено исключительно в результате экспериментального исследования, что соответствует общепринятой практике в области распознавания образов.

В последующих разделах главы рассматриваются конкретные методы формирования такого созвездия иероглифа и обосновывается их применимость для задач поиска в древнекитайских рукописных архивах.



\section{Построение медиальной оси и скелетного метаграфа}\label{sec:ch3/sec2}
% Из ISPRS
Предложенное в данной работе решение основано на построении морфологической модели иероглифа в виде геометрического графа. Такой граф далее будем называть \textit{метаграфом иероглифа}. Метаграф служит компактным и структурно устойчивым представлением формы иероглифа и используется на последующих этапах для построения созвездий и их сопоставления.

Построение метаграфа осуществляется на основе преобразования исходного цифрового изображения иероглифа и опирается на медиальное представление формы \cite{siddiqi2008medial}. Медиальное представление состоит из двух взаимосвязанных компонентов: медиальной оси и радиальной функции. Медиальная ось (или скелет) определяется как множество центров всех максимальных окружностей, вписанных в фигуру, а радиальная функция в каждой точке скелета равна радиусу соответствующей окружности. Ключевыми преимуществами данной модели являются строгое математическое определение, высокая точность описания формы и вычислительная эффективность алгоритмов построения.

Общая последовательность действий при построении метаграфа представлена на рисунке \cref{fig:sec3_construction_metagraph}. Она включает в себя следующие шаги:
\begin{enumerate}
    \item Бинаризация исходного изображения иероглифа. Это процесс преобразования исходного изображения в черное-белое растровое изображение, которое используется в качестве входных данных для последующих морфологических операций и анализа формы.
    \item Аппроксимация бинарного изображения иероглифа многоугольной фигурой. Выбор многоугольных фигур для аппроксимации формы иероглифов обусловлен возможностью использования эффективных алгоритмов вычислительной геометрии для построения их скелетов на основе обобщенных диаграмм Вороного \cite{siddiqi2008medial}.
    \item Построение непрерывного скелета многоугольной фигуры. Скелет — это множество точек центров максимальных вписанных в фигуру окружностей. Получение непрерывного скелета осуществляется посредством построения обобщенной диаграммы Вороного многоугольной фигуры и последующей фильтрации рёбер диаграммы Вороного.
    \item Регуляризация (стрижка) скелета, состоящая в последовательном отсечении части ребер скелета. Этот процесс называется стрижкой (pruning). Использование предлагаемой непрерывной модели позволяет получить математически строгий критерий стрижки, основанный на концепции силуэта скелетного подграфа \cite{LM2016continuous}.
\end{enumerate}

\begin{figure}[ht]
    \centerfloat{
        \includegraphics[width=0.7\linewidth]{sec3_construction_metagraph.jpg}
    }
    \caption{Последовательность операций при построении метаграфа иероглифов на основе преобразования цифрового изображения.}\label{fig:sec3_construction_metagraph}
\end{figure}

\begin{figure}[ht]
    \centerfloat{
        \hfill
        \subcaptionbox{\label{fig:sec3_metagraph_a}}{%
            \includegraphics[width=0.3\linewidth]{sec3_metagraph_a.jpg}}
        \hfill
        \subcaptionbox{\label{fig:sec3_metagraph_b}}{%
            \includegraphics[width=0.3\linewidth]{sec3_metagraph_b.jpg}}
        \hfill
        \subcaptionbox{\label{fig:sec3_metagraph_c}}{%
            \includegraphics[width=0.3\linewidth]{sec3_metagraph_c.jpg}}
        \hfill
        \par\bigskip
        \hfill
        \subcaptionbox{\label{fig:sec3_metagraph_d}}{%
            \includegraphics[width=0.3\linewidth]{sec3_metagraph_d.jpg}}
        \hfill
        \subcaptionbox{\label{fig:sec3_metagraph_e}}{%
            \includegraphics[width=0.3\linewidth]{sec3_metagraph_e.jpg}}
        \hfill
        \subcaptionbox{\label{fig:sec3_metagraph_f}}{%
            \includegraphics[width=0.3\linewidth]{sec3_metagraph_f.jpg}}
        \par\bigskip
        \hfill
        \subcaptionbox{\label{fig:sec3_metagraph_g}}{%
            \includegraphics[width=0.3\linewidth]{sec3_metagraph_g.jpg}}
        \hfill
        \subcaptionbox{\label{fig:sec3_metagraph_h}}{%
            \includegraphics[width=0.3\linewidth]{sec3_metagraph_h.jpg}}
        \hfill
        \subcaptionbox{\label{fig:sec3_metagraph_i}}{%
            \includegraphics[width=0.3\linewidth]{sec3_metagraph_i.jpg}}
        \hfill
    }
    \caption[]{Иллюстрационный пример построения метаграфа иероглифа для одного цифрового изображения.}\label{fig:sec3_metagraph}
\end{figure}

Выбор многоугольных фигур для аппроксимации формы иероглифов обусловлен возможностью использования эффективных алгоритмов вычислительной геометрии для построения их медиального представления на основе диаграмм Вороного.

Аппроксимирующая многоугольная фигура может состоять из нескольких компонент связности, каждая из которых представляет собой многоугольник с многоугольными отверстиями (рисунок \cref{fig:sec3_metagraph_e}). Аппроксимация иероглифа многоугольной фигурой показана на рисунках \cref{fig:sec3_metagraph_a}-\cref{fig:sec3_metagraph_d}. Для исходного растрового изображения иероглифа (рисунок \cref{fig:sec3_metagraph_b}) необходимо выполнить поиск, трассировку и аппроксимацию всех граничных контуров фигуры \cite{mestetskiy2008binary}.

Построенные многоугольники для всех граничных контуров образуют аппроксимирующую многоугольную фигуру для изображения иероглифа (рисунок \cref{fig:sec3_metagraph_e}).

Следующим шагом предлагаемого подхода является построение внутренней диаграммы Вороного (ДВ) полученной многоугольной фигуры (рисунок \cref{fig:sec3_metagraph_f}). Для этого граничные многоугольники фигуры разбиваются на подмножества, называемые сайтами. Сайтами являются все вершины и все стороны граничных многоугольников. Внутренняя ДВ многоугольной фигуры представляет собой разбиение множества точек фигуры на локусы – подмножества точек, ближайших к одному из сайтов фигуры. Для построения внутренней ДВ мы используем алгоритм \cite{mestetskiy2024constructing}.

Локусы ДВ пересекаются в точках их границ. На рисунке \cref{fig:sec3_metagraph_f} эти пересечения показаны красными линиями. Этот набор линий называется ребрами ДВ. Его можно рассматривать как плоский геометрический граф. Скелет – это подграф ДВ, в него входят почти все ребра ДВ, кроме ребер, инцидентных локусам вогнутых вершин многоугольной фигуры. Таким образом, скелет можно получить на основе отсечения этих ребер ДВ (рисунок \cref{fig:sec3_metagraph_f}).

Далее строится метаграф иероглифа, представляющий собой подграф скелета. Процесс выделения подграфа сводится к последовательному отсечению части ребер скелета. Этот процесс называется стрижкой. Использование предлагаемой непрерывной модели позволяет получить математически строгий критерий стрижки, основанный на концепции силуэта скелетного подграфа \cite{mestetskiy2024constructing}. Силуэтом скелетного подграфа называется объединение всех вписанных кругов фигуры с центрами на вершинах и рёбрах подграфа. Общая идея стрижки состоит в том, чтобы выделить в скелете минимальный подграф, у которого силуэт отличается от многоугольной фигуры не более чем на заданную пороговую величину в метрике Хаусдорфа. Необходимо последовательно отсекать те терминальные ребра подграфа, при удалении которых расстояние Хаусдорфа между силуэтом подграфа и многоугольной фигурой не превышает заданного порога. В качестве порогового значения для изображений иероглифов может быть выбран размер пикселя. Пример полученного в результате стрижки подграфа показан на рисунках \cref{fig:sec3_metagraph_g}-\cref{fig:sec3_metagraph_h}.

Использование медиального представления в качестве основы для построения метаграфа принципиально отличается от подходов, основанных на извлечении обучаемых признаков, таких как сверточные нейронные сети. В отличие от них, медиальное представление обеспечивает геометрическую интерпретируемость всех элементов модели, инвариантность к локальным искажениям контура и устойчивость к изменениям толщины штрихов.

Построенный метаграф иероглифа представляет собой компактное графовое описание его формы, в котором узлы и рёбра обладают как геометрической, так и топологической интерпретацией. Такое представление является удобной основой для дальнейшего выделения локальных структурных элементов — созвездий, а также для формализации задачи их сопоставления между различными иероглифами.

\section{Получение описания иероглифа <<созвездием>>}\label{sec:ch3/sec3}
На предыдущем этапе был построен метаграф иероглифа, представляющий собой регуляризованный подграф медиального скелета и обладающий устойчивыми геометрическими и топологическими свойствами. Однако сам по себе метаграф является достаточно сложным объектом, прямое сопоставление которого между различными иероглифами затруднено из-за вариативности формы и большого количества входящих в него вершин и ребер.

В связи с этим возникает необходимость перехода от глобального графового представления к более компактному набору локальных признаков, сохраняющих существенную структурную информацию. В данной работе такой набор локальных признаков формируется в виде так называемого \emph{созвездия}, представляющего собой совокупность характерных точек метаграфа и их взаимного расположения.

Формирование точечных признаков осуществляется на основе анализа вершин подграфа скелета, полученного в результате стрижки, описанной в предыдущем подразделе. В общем случае вершины скелетного подграфа могут иметь степень 1, 2 или 3. Естественным и на первый взгляд нетривиальным выбором является включение в описание только терминальных и разветвлённых вершин, то есть вершин степени 1 и 3, которые интуитивно соответствуют окончаниям штрихов и точкам их ветвления.

Однако на практике оказывается, что использование только вершин степени 1 и 3 является недостаточным для получения адекватного и устойчивого описания иероглифа. На рисунке \cref{fig:sec3_example_1_3_not_enough} изображена локальная структура скелетного графа, для которой принципиально важно наличие вершины 2 степени. Потеря таких вершин приводит к утрате информации о кривизне соответствующих элементов (ломанная линия $A - B - C - D$). В ряде таких случаев такое описание теряет существенную информацию о протяжённых участках штрихов и их относительной конфигурации, что негативно сказывается на последующем этапе сопоставления. Пример иероглифа, для которого данного набора вершин оказывается недостаточно, приведены на рисунке \cref{fig:sec3_1_3_nodes_not_enough}. При отсутствии вершин второй степени, выделенных красным цветом, утрачивается информация о локальной кривизне штрихов иероглифа. Кроме того, в случае наличия в структуре иероглифа внутренних отверстий соответствующие локальные топологические особенности без учёта вершин второй степени не выявляются и, фактически, игнорируются.

\begin{figure}[ht]
    \centerfloat{
        \includegraphics[width=0.9\linewidth]{sec3_example_1_3_not_enough.jpg}
    }
    \caption{Пример локальной структуры скелетного графа, для которого важны вершины 2 степени.}\label{fig:sec3_example_1_3_not_enough}
\end{figure}

\begin{figure}[ht]
    \centerfloat{
        \includegraphics[width=0.8\linewidth]{sec3_1_3_nodes_not_enough.jpg}
    }
    \caption{Пример точечного описания иероглифа. Слева показано исходное изображение иероглифа в бинарном представлении, справа — его скелетный граф с соответствующим набором точек созвездия, обозначенных цветными точками. Зелеными обозначаем вершины 1 степени, синими - вершины 3 степени, а красными - вершины 2 степени.}\label{fig:sec3_1_3_nodes_not_enough}
\end{figure}

Для получения более устойчивого и информативного описания формы иероглифа в данной работе предлагается расширить набор используемых характерных точек метаграфа. Пусть метаграф иероглифа задаётся в виде графа
\(G = (V, E)\),
где каждой вершине и каждой точке ребра сопоставлено её геометрическое положение на плоскости. Множество вершин метаграфа формируется на основе двух правил:
\begin{enumerate}
    \item Вершинами метаграфа считаются все вершины 1 и 3 степени этого подграфа, то есть концевые и разветвлённые узлы скелетных ветвей.
    \item Некоторое подмножество вершин 2 степени также объявляется вершинами метаграфа для учёта данных о форме рёбер метаграфа. Это вершины максимальной кривизны в цепях рёбер, соединяющих вершины 1 и 3 степени.
\end{enumerate}

На основе полученного множества точек формируется описание иероглифа в виде \emph{созвездия}. Под созвездием будем понимать конечное множество точек \(\mathcal{C} = \{p_1, p_2, \dots, p_N\}\),
где каждая точка \(p_i \in V\) характеризуется своим положением в нормализованной системе координат. Созвездие, в отличие от исходного метаграфа, не содержит явной информации о связности между точками. Однако относительное расположение точек в пространстве и их локальные характеристики неявно кодируют структуру иероглифа и оказываются достаточными для последующего сопоставления.

В следующих подразделах подробно рассматривается критерий выбора вершин степени~2, основанный на анализе кривизны рёбер скелета. Кроме того, описывается процедура нормализации, направленная на устранение влияния масштаба и обеспечение сопоставимости созвездий, полученных из различных иероглифов. Наконец, отдельное внимание уделяется задаче удаления аномальных точек и структур, возникающих в результате шумов скелетизации и особенностей исходных изображений, в том числе в виде малых или слабо связанных компонент.

\subsection{Выбор критических точек с максимальной кривизной}\label{sec:ch3/sec3/sub1}
% 1. определение максимальной кривизны 
% 2. влияние выбора угла для данной задачи (+ картинка)
% 3. Смерживание близких по расстоянию вершин для снижения размерности задачи

Как было отмечено ранее, вершины степени~2 составляют значительную часть скелетного подграфа и соответствуют протяжённым участкам штрихов иероглифа. Включение всех таких вершин в состав созвездия приводит к избыточному описанию формы и существенному росту размерности задачи сопоставления. В связи с этим в данной работе предлагается отбирать лишь те вершины степени~2, которые соответствуют локально значимым изменениям геометрии штрихов.

Пусть метаграф иероглифа задаётся неориентированным графом \(G = (V, E)\), вложенным в евклидово пространство \(\mathbb{R}^2\). Рассмотрим вершину \(v \in V\) степени~2, инцидентную рёбрам
\(e_1 = (v, v_1)\) и \(e_2 = (v, v_2)\). Каждому ребру сопоставляется вектор направления
\(\mathbf{d}_1 = \overrightarrow{v v_1}\) и \(\mathbf{d}_2 = \overrightarrow{v v_2}\). Локальная кривизна в вершине \(v\) оценивается через угол \(\theta(v)\) между этими векторами:
\[
    \theta(v) =
    \arccos
    \frac{\langle \mathbf{d}_1, \mathbf{d}_2 \rangle}
    {\|\mathbf{d}_1\| \, \|\mathbf{d}_2\|}.
\]

Вершина \(v\) степени~2 считается \emph{критической} и сохраняется в составе созвездия, если выполняется условие
\[
    \theta(v) < \theta_{\mathrm{thr}},
\]
где \(\theta_{\mathrm{thr}}\) — заданное пороговое значение. Такие вершины соответствуют локальным максимумам кривизны и отражают существенные изменения направления штриха.

Если же для вершины степени~2 выполняется условие \(\theta(v) \ge \theta_{\mathrm{thr}}\), то она интерпретируется как принадлежащая почти прямолинейному участку скелета точка с низкой кривизной и не вносит значимой информации о форме. В этом случае вершина удаляется из графа, а её инцидентные рёбра \((v_1, v)\) и \((v, v_2)\) заменяются одним ребром \((v_1, v_2)\), при условии \(v_1 \neq v_2\). Данная операция сохраняет топологическую связность скелетного подграфа и приводит к упрощению структуры метаграфа без потери существенных геометрических особенностей.

Пороговое значение угла \(\theta_{\mathrm{thr}}\) является гиперпараметром и выбирается равным \(135^\circ\) в ходе проведения экспериментов. Как показано на рисунке~\cref{fig:XXX}, при меньших значениях порога в созвездие включается избыточное число вершин степени~2, соответствующих слабо изогнутым участкам скелета, что приводит к переусложнению описания. При увеличении порогового значения, напротив, в ряде случаев точки максимальной кривизны не выявляются вовсе, вследствие чего теряется существенная информация, необходимая для адекватного описания формы иероглифа.

Помимо отбора вершин по кривизне, вводится дополнительная операция уменьшения плотности точек в областях с избыточной детализацией. Пусть \(\mathcal{C} = \{p_i\}\) — множество выбранных точек созвездия. Для заданного расстояния \(\varepsilon > 0\) все пары точек \(p_i, p_j \in \mathcal{C}\), удовлетворяющие условию \(\|p_i - p_j\| < \varepsilon\), объединяются в одну представительную точку, положение которой определяется, например, как их центр масс. Здесь \(\varepsilon\) - также гиперпараметр алгоритма. Данная операция позволяет уменьшить общее число элементов созвездия, снизить вычислительную сложность последующего сопоставления и повысить устойчивость описания к локальным шумам.

\subsection{Метод нормализации}\label{sec:ch3/sec3/sub2}
% Из PRIA
Задача определения сходства и различия иероглифов основана на идее сравнения положения точек, входящих в созвездия иероглифов. Для корректного сравнения созвездия должны быть нормализованы, т.е. преобразованы в определённый стандартный формат, что позволяет сравнивать их между собой. Нормализация созвездий включает два преобразования: \textit{центрирование} и \textit{масштабирование}. Центрирование преобразует созвездие к единой системе координат, а масштабирование приводит созвездия к единому масштабу. Нормализация основывается на определении понятия центра и отклонения для фигуры иероглифа. При описании иероглифа в виде созвездия существует простой приемлемый по точности способ, состоящий в распределении некоторой «массы» по точкам созвездия, причем масса всех точек одинакова.

Центрирование основано на вычислении центра массы созвездия. Центр массы (или геометрический центр) вычисляется как среднее значение координат всех вершин, входящих в точечное представление иероглифа. Пусть $P=\{p_i=(x_i,y_i)|i=0,1,\ldots,N\}$ – координаты точек созвездия, тогда центр массы $c=(x_c, y_c)$ вычисляется следующим образом:
\[
    c=(x_c, y_c)=\frac{1}{N} \sum_{i=1}^{N}p_i = (\sum_{i=1}^{N}x_i,\sum_{i=1}^{N}y_i)\]
Далее масштабирование осуществляется относительно полученного центра массы. Преобразование состоит в сдвиге и умножении координат точек созвездий на коэффициент, определяемый на основе вычисления среднеквадратичного отклонения $\sigma$ точек созвездия от центра массы
\[
    \sigma = \sqrt{\frac{1}{N}\sum_{i=1}^{N}\|p_i-c\|^2} = \sqrt{\frac{1}{N}\sum_{i=1}^{N}(x_i-x_c)^2+(y_i-y_c)^2}
\]
Координаты точек преобразуются таким образом, чтобы созвездия имели примерно одинаковую пространственную протяженность независимо от исходного размера, т.е. отклонение равнялось единице. При масштабировании созвездия координаты пересчитываются по формуле:
\[
    \tilde{p_i} = \frac{p_i-c}{\sigma} = (\frac{x_i-x_c}{\sigma},\frac{y_i-y_c}{\sigma})
\]

\subsection{Удаление аномальных связанных компонент}\label{sec:ch3/sec3/sub3}

В процессе построения скелетного подграфа и последующего формирования созвездия на практике могут возникать аномальные структуры, обусловленные неточностями сегментации изображения, шумами бинаризации и особенностями морфологической скелетизации. Примеры аномальных структур для иероглифов представлены на рисунке \cref{fig:sec3_anomality}. Как правило, такие структуры проявляются в виде малых изолированных связанных компонент, которые не несут существенной информации о форме иероглифа, но при этом увеличивают размерность описания, нарушают нормализацию созвездия и могут негативно влиять на устойчивость процедуры сопоставления.

\begin{figure}[ht]
    \centerfloat{
        \hfill
        \subcaptionbox{\label{fig:sec3_anomality_1}}{%
            \includegraphics[width=0.3\linewidth]{sec3_anomality_1.jpg}}
        \hfill
        \subcaptionbox{\label{fig:sec3_anomality_2}}{%
            \includegraphics[width=0.3\linewidth]{sec3_anomality_2.jpg}}
        \hfill
        \subcaptionbox{\label{fig:sec3_anomality_3}}{%
            \includegraphics[width=0.3\linewidth]{sec3_anomality_3.jpg}}
        \hfill
        \par\bigskip
        \hfill
        \subcaptionbox{\label{fig:sec3_anomality_4}}{%
            \includegraphics[width=0.3\linewidth]{sec3_anomality_4.jpg}}
        \hfill
        \subcaptionbox{\label{fig:sec3_anomality_5}}{%
            \includegraphics[width=0.3\linewidth]{sec3_anomality_5.jpg}}
        \hfill
        \subcaptionbox{\label{fig:sec3_anomality_6}}{%
            \includegraphics[width=0.3\linewidth]{sec3_anomality_6.jpg}}
        \par\bigskip
        \hfill
    }
    \caption[]{Иллюстрационные примеры аномальных структур, выделенных красными рамками.}\label{fig:sec3_anomality}
\end{figure}

Пусть метаграф иероглифа задаётся графом
\(G = (V, E)\),
а множество его связных компонент обозначим через
\(\{\mathcal{C}_k\}\).
В рамках данной работы рассматриваются простые эвристические критерии
выявления и удаления аномальных компонент,
основанные на их мощности и геометрических характеристиках.

Во-первых, все связные компоненты, состоящие из одной вершины,
то есть удовлетворяющие условию
\(|\mathcal{C}_k| = 1\),
рассматриваются как шумовые и удаляются.
Такие одиночные вершины, как правило,
соответствуют артефактам скелетизации
и не отражают устойчивых элементов структуры иероглифа.

Во-вторых, рассматриваются связные компоненты,
состоящие ровно из двух вершин:
\(|\mathcal{C}_k| = 2\).
Пусть координаты этих вершин равны
\(p_1, p_2 \in \mathbb{R}^2\).
Если евклидово расстояние между ними удовлетворяет условию
\begin{equation}
    \|p_1 - p_2\| < \varepsilon,
    \label{eq:3.1}
\end{equation}
где \(\varepsilon > 0\) — заданный порог,
то такая компонента интерпретируется
как локальный шумовой фрагмент и также удаляется из метаграфа.
Данный критерий позволяет отсеять короткие и изолированные сегменты,
возникающие в результате локальных искажений границы фигуры.

Используемая процедура удаления аномальных компонент
имеет линейную вычислительную сложность
относительно числа вершин графа
и легко интегрируется в общий конвейер построения созвездий.
При этом она существенно снижает число ложных или
малозначимых точек,
повышая стабильность и воспроизводимость
последующего этапа сопоставления.

Следует отметить, что предложенные критерии
не являются единственно возможными.
В качестве альтернативных или дополнительных правил
могут рассматриваться ограничения
на суммарную длину рёбер компоненты,
её удалённость от основной части метаграфа,
а также соотношение между геометрическим размером компоненты
и масштабом всего иероглифа.
Тем не менее, в рамках данной работы
используется описанная простая эвристика,
которая на практике демонстрирует
хороший компромисс между вычислительной эффективностью
и качеством фильтрации шумов.

К возможным ограничениям данного подхода
следует отнести риск удаления редких,
но структурно значимых мелких элементов,
а также зависимость результата от выбора порогового параметра \(\varepsilon\) в (\ref{eq:3.1}).
Однако, как показывают эксперименты,
в большинстве рассматриваемых случаев
данная процедура не приводит к потере
существенной информации о форме иероглифа,
при этом заметно повышая устойчивость
всей системы в целом.


\section{Выводы к главе \cref{ch:ch3}}

В рамках данной главы была обоснована целесообразность использования непрерывных морфологических моделей и медиального представления формы для анализа иероглифических изображений. Показано, что медиальная ось, полученная на основе диаграммы Вороного бинарного изображения, позволяет компактно и устойчиво описывать внутреннюю структуру иероглифа, сохраняя его ключевые топологические и геометрические свойства.

Предложен метод построения скелетного метаграфа, в котором узлы и рёбра отражают критические элементы структуры символа, такие как точки ветвления, окончания штрихов и характерные изгибы. Данный метаграф служит промежуточным представлением между исходным изображением и конечным набором признаков, обеспечивая структурную интерпретируемость и устойчивость к локальным искажениям формы.

Разработан подход к формированию признакового описания иероглифа в виде конечного множества точек, интерпретируемого как «созвездие» характерных элементов формы. Особое внимание уделено выбору критических точек на основе анализа локальной кривизны медиальной оси, что позволяет отбирать наиболее информативные элементы структуры иероглифа и устранять избыточные или малоустойчивые детали. Рассмотрены методы нормализации полученного точечного описания, направленные на устранение влияния масштаба, положения и глобальных деформаций изображения. Дополнительно предложен алгоритм удаления аномальных связанных компонент, возникающих вследствие шумов, дефектов сканирования или деградации рукописного материала, что повышает устойчивость и воспроизводимость признакового описания.

Таким образом, в главе 3 сформирована полная и последовательная алгоритмическая процедура построения точечного признакового представления иероглифических изображений — от исходного бинарного изображения до нормализованного набора критических точек. Полученное представление является компактным, интерпретируемым и вычислительно эффективным, а также хорошо приспособленным для последующего сопоставления и вычисления меры сходства между иероглифами. В дальнейшем полученные точечные признаки используются при сопоставлении и поиске иероглифов, обеспечивая возможность сравнения их структур по аналогии с методом «созвездий».

\FloatBarrier
