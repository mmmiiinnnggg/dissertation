\chapter{Методы построения меры сходства изображений иероглифических текстов}\label{ch:ch3}

TODO: Оценка вычислительной сложности
% Из ISPRS

Мера сходства и различия иероглифов строится на основе сравнения их метаграфов. С этой точки зрения мы видим одно из достоинств предложенного метода: простое, но эффективное построение признаков, по сравнению с методами, основанными на глубоком обучении, которые требуют относительно большого количества ресурсов для применения. Кроме того, весь процесс является прозрачным и интерпретируемым, что дает больше возможностей для манипулирования генерацией признаков.

Сложность сравнения состоит в том, что метаграфы изображений одного и того же иероглифа часто оказываются неизоморфными графами. Это происходит из-за естественных различий в начертании рукописных символов, деформации древних рукописей вследствие длительного хранения, а также искажений при сканировании документов. Деформации и искажения выражаются в том, что на полученных цифровых изображениях некоторые штрихи слипаются, либо в них появляются разрывы. Поэтому топологический критерий сходства метаграфов недостаточен сравнения иероглифов. Предлагаемая мера близости созвездий определяется на основе построения наилучшего паросочетания точек между точками созвездий. Критерий качества паросочетания может быть выбран различными способами. Мы рассматриваем два варианта такого выбора. Оба варианта критерия приводят к оптимизационной задаче назначений.

\section{Задача о назначениях и линейное программирование}\label{sec:ch3/sec0}
TODO: обзор в математическую основу

\section{Модель минимальной стоимости}\label{sec:ch3/sect1}
Первая модель ищет паросочетание, минимизирующее различия между созвездиями. Она состоит в выборе для каждой точки одного созвездия хотя бы одной соответствующей точки в другом созвездии. Задача состоит в выборе из всех допустимых вариантов соответствия точек такого, при котором сумма расстояний между соответствующими парами точек будет минимальна. Такой выбор осуществляется на основе оптимизационной модели задачи о назначениях.

Формально, пусть $n,m$ – число точек в двух созвездиях. Обозначим $X=\{x_{ij}\}$ – бинарные переменные, указывающие соответствие между $i$-ой точкой первого созвездия и $j$-ой точкой второго созвездия. Если $x_{ij}=1$, то соответствие между этими точками установлено, если $x_{ij}=0$, то нет. Пусть $\lambda_{ij}$ – коэффициенты стоимости для паросочетаний точек $(i,j)$. Коэффициенты определяют веса для соответствия точек созвездий, представляющих, например, евклидово расстояние между ними.

Оптимизационная задача о назначениях имеет следующий вид:
\[
    \min f(X) = \sum_{i=1}^{n}\sum_{j=1}^{m}\lambda_{ij}x_{ij}, \quad X =\{x_{ij}\} \in \{0,1\}, \sum_{i=1}^{n}x_{ij}\ge 1, \sum_{j=1}^{m}x_{ij}\ge 1.
\]
Последние два ограничения на выбор паросочетаний гарантируют, что у каждой точки одного созвездия есть как минимум одна соответствующая точка в другом созвездии, чтобы каждая точка была учтена при подсчете меры близости. На основе решения этой оптимизационной задачи оценивается мера различия двух метаграфов. Чем меньше значение $f(X)$, тем меньше различаются метаграфы, тем больше сходство иероглифов.

\subsection{Проверка соединения соответвующих вершин через ребра (connectivity checking)}\label{subsec:ch3/sect3/sub1}
TODO:
\subsection{Минимальная стоимость по ребрам}\label{subsec:ch3/sect3/sub2}
TODO:

\section{Модель максимального количества близких пар}\label{sec:ch3/sec2}
Другая модель максимизирует меру сходства созвездий. Она состоит в выборе максимального числа допустимых паросочетаний при условии, что допустимыми парами является только те, у которых евклидово расстояние $\lambda_{ij}$ между точками не превосходит заданного порога, и при этом каждая точка может входить не более чем в одну пару. Пусть коэффициент $\mu_{ij}$ – индикатор того, что расстояние $\lambda_{ij}$ между $i$-ой точкой первого созвездия и $j$-ой точкой второго созвездия не превышает порога $p$, т.е. $\mu_{ij}=\mathbb{I}[\lambda_{ij}\le p]$.

Здесь тоже получается задача о назначениях, которая может быть записана в следующем виде:
\[
    \max f(X) = \sum_{i=1}^{n}\sum_{j=1}^{m}\mu_{ij}x_{ij}, \quad X =\{x_{ij}\} \in \{0,1\}, \sum_{i=1}^{n}x_{ij}\le 1, \sum_{j=1}^{m}x_{ij}\le 1.
\]
Величина $\frac{f(X)}{mn}$ , полученная на основе решения этой задачи, используется как оценка сходства между созвездиями двух иероглифов. Для решения задачи о назначениях в обеих моделях мы используем алгоритм, реализованный в библиотеке \verb|PuLP| \cite{mitchell2011pulp}.

При решении задачи поиска иероглифов в файле по заданному запросу вычисляется мера сходства и различия запроса со всеми иероглифами файла. В случае использования первой модели составляется ранжированный список иероглифов из файла по возрастанию различий между ними и запросом. При использовании второй модели получается ранжированный список по убыванию сходства запроса и иероглифов из файла. Первые по порядку иероглифы в этих списках будут рассматриваться как кандидаты для решения задачи поиска. Они предъявляются исследователю в качестве вариантов решения задачи.


\section{Сравнение и обсуждение}

\section{Выводы к главе \cref{ch:ch3}}
\clearpage
